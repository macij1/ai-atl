The learning algorithm $\lstar$ was introduced more than thirty years
ago and has been intensively extended to many types of models in
following years. This algorithm continues to attract the attention of
many researchers~\cite{Angluin17,SchroderKMW17,Moerman17}.

We designed a learning algorithm for a class of languages over
infinite alphabet; more precisely, we have considered nominal regular
languages with binders~\cite{Kurz0T12,Kurz0T13}.
%
We have tackled the finitary representations of the alphabets, words
and automata for retaining the basic scheme and ideas of
$\lstar$. Hence, we revised and added definitions for the nominal
words and automata.
%
Further, accounting for names and the allocation and deallocation
operations, we revised the data structures and notions in
$\lstar$. Accordingly, we have proposed the learning algorithm,
$\nlstar$, to stress the progress of learning a nominal language with
binders. We have proved the correctness and analysed the complexities
of $\nlstar$.

As for \lstar, a key factor for the effectiveness of and \nlstar is
the selection of counterexamples.
%
%
Due to possibly infinite number of candidate counterexamples, the
selection of the counterexamples is non-deterministic in \nlstar.
%
We are developing an implementation of $\nlstar$ to study effective
mechanisms to resolve this non-determinism.
%
Interestingly, the rich structure of nominal automata offer different
directions to solve this problem.
%
In fact, we started to investigate this issue and defined two
different strategies used by the teacher to generate counterexamples
based on the \quo{size} of the counterexamples and on the preference
of the teacher for counterexamples with maximal or minimal number of
fresh names.
%
Initial experiments show how the strategies impact on the
convergence of \nlstar.
%
This immediately suggest that \nlstar could be improved by designing
different strategies to generate \quo{better} counterexamples, that is
counterexamples that allow the learner to learn \quo{more quickly}.
% \begin{new}
%   The use of symbol \quo{P} allows us to attain a significant
%   improvement on the performance of the learning process (as hinted
%   from Table~\ref{tb:comparison}). We are now trying to find out the
%   exact factors which have been affected. In the future, we conclude a
%   new strategy within the \quo{P} symbol.
% \end{new}
%%% Local Variables:
%%% mode: latex
%%% TeX-master: "main"
%%% End:
