\begin{figure}[t]
    \centering
    \includegraphics[width=\textwidth]{assets/llama3_architecture.pdf}
    \caption{\textbf{Illustration of the compositional approach to adding multimodal capabilities to Llama 3 that we study in this paper.} This approach leads to a multimodal model that is trained in five stages: \textbf{(1)} language model pre-training, \textbf{(2)} multi-modal encoder pre-training, \textbf{(3)} vision adapter training, \textbf{(4)} model finetuning, and \textbf{(5)} speech adapter training.}
    \label{sph:fig:multimodal_model_overview}
\end{figure}

\section{Vision Experiments}
\label{section:vision}

We perform a series of experiments in which we incorporate visual-recognition capabilities into \llamathree via a compositional approach that consists of two main stages.
First, we compose a pre-trained image encoder \citep{xu2023demystifying} and the pre-trained language model by introducing and training a set of cross-attention layers between the two models \citep{alayrac2022flamingo} on a large number of image-text pairs.
This leads to the model illustrated in Figure~\ref{sph:fig:multimodal_model_overview}.
Second, we introduce temporal aggregator layers and additional video cross-attention layers that operate on a large collection of video-text pairs to learn the model to recognize and process temporal information from videos.

A compositional approach to foundation model development has several advantages: \textbf{(1)} it enables us to parallelize the development of the vision and language modeling capabilities; \textbf{(2)} it circumvents complexities of joint pre-training on visual and language data that stem from tokenization of visual data, differences in background perplexities of tokens originating from different modalities, and contention between modalities; \textbf{(3)} it guarantees that model performance on text-only tasks is not affected by the introduction of visual-recognition capabilities, and \textbf{(4)} the cross-attention architecture ensures that we do not have to expend compute passing full-resolution images through the increasingly LLM backbones (specifically, the feed-forward networks in each transformer layer), making it more efficient during inference.
We note that our multimodal models are still under development and not yet ready for release.

Before presenting the results of our experiments in Section~\ref{section:results_image_recognition} and~\ref{section:results_video_recognition}, we describe the data we used to train visual recognition capabilities, the model architecture of the vision components, how we scale training of those components, and our pre-training and post-training recipes.

\subsection{Data}
\subsection{Data}
\subsection{Data}
\input{speech/asr/data.tex}
\input{speech/tts/data.tex}

\subsection{Data}
\input{speech/asr/data.tex}
\input{speech/tts/data.tex}


\subsection{Data}
\subsection{Data}
\input{speech/asr/data.tex}
\input{speech/tts/data.tex}

\subsection{Data}
\input{speech/asr/data.tex}
\input{speech/tts/data.tex}



\subsection{Model Architecture}

\subsection{Model Architecture}

\subsection{Model Architecture}

\input{speech/asr/model_architecture.tex}
\input{speech/tts/model_architecture.tex}

\subsection{Model Architecture}

\input{speech/asr/model_architecture.tex}
\input{speech/tts/model_architecture.tex}


\subsection{Model Architecture}

\subsection{Model Architecture}

\input{speech/asr/model_architecture.tex}
\input{speech/tts/model_architecture.tex}

\subsection{Model Architecture}

\input{speech/asr/model_architecture.tex}
\input{speech/tts/model_architecture.tex}



\newcommand{\cq}[1]{\textcolor{red}{\{CQ: #1\}}} %

\subsection{Infrastructure, Scaling, and Efficiency}
\label{section:pretraining_model_scaling}
We describe our hardware and infrastructure that powered \llamathree 405B pre-training at scale and discuss several optimizations that leads to improvements in training efficiency.

\subsubsection{Training Infrastructure}
The Llama 1 and 2 models were trained on Meta's AI Research SuperCluster~\citep{Lee22RSC}. As we scaled further, the training for Llama 3 was migrated to Meta's production clusters~\citep{lee2024building}.%
This setup optimizes for production-grade reliability, which is essential as we scale up training.

\textbf{Compute.}
\llamathree 405B is trained on up to 16K H100 GPUs, each running at 700W TDP with 80GB HBM3, using Meta's Grand Teton AI server platform~\citep{various2022grandteton}. Each server is equipped with eight GPUs and two CPUs. Within a server, the eight GPUs are connected via NVLink. Training jobs are scheduled using MAST~\citep{choudhury2024mast}, Meta's global-scale training scheduler.

\textbf{Storage.} 
Tectonic~\citep{pan2021tectonicfs}, Meta's general-purpose distributed file system, is used to build a storage fabric~\citep{battey2024storage} for Llama 3 pre-training. It offers 240 PB of storage out of 7,500 servers equipped with SSDs, and supports a sustainable throughput of 2 TB/s and a peak throughput of 7 TB/s. A major challenge is supporting the highly bursty checkpoint writes that saturate the storage fabric for short durations. Checkpointing saves each GPU’s model state, ranging from 1 MB to 4 GB per GPU, for recovery and debugging. We aim to minimize GPU pause time during checkpointing and increase checkpoint frequency to reduce the amount of lost work after a recovery. 

\textbf{Network.}
Llama 3 405B used RDMA over Converged Ethernet (RoCE) fabric based on the Arista 7800 and Minipack2 Open Compute Project\footnote{Open Compute Project: \url{https://www.opencompute.org/}} OCP rack switches. Smaller models in the Llama 3 family were trained using Nvidia Quantum2 Infiniband fabric. Both RoCE and Infiniband clusters leverage 400 Gbps interconnects between GPUs.  Despite the underlying network technology differences between these clusters, we tune both of them to provide equivalent performance for these large training workloads. We elaborate further on our RoCE network since we fully own its design.
\begin{itemize}

    \item \textbf{Network topology.} Our RoCE-based AI cluster comprises 24K GPUs\footnote{Note that we use only up to 16K of these 24K GPUs for Llama 3 pre-training.} connected by a three-layer Clos network~\citep{lee2024building}. At the bottom layer, each rack hosts 16 GPUs split between two servers and connected by a single Minipack2 top-of-the-rack (ToR) switch. In the middle layer, 192 such racks are connected by Cluster Switches to form a pod of 3,072 GPUs with full bisection bandwidth, ensuring no oversubscription. At the top layer, eight such pods within the same datacenter building are connected via Aggregation Switches to form a cluster of 24K GPUs. However, network connectivity at the aggregation layer does not maintain full bisection bandwidth and instead has an oversubscription ratio of 1:7. Our model parallelism methods (see Section~\ref{section:4D-parallelism}) and training job scheduler~\citep{choudhury2024mast} are all optimized to be aware of network topology, aiming to minimize network communication across pods.
    
    \item \textbf{Load balancing.} LLM training produces fat network flows that are hard to load balance across all available network paths using traditional methods such as Equal-Cost Multi-Path (ECMP) routing. To address this challenge, we employ two techniques. First, our collective library creates 16 network flows between two GPUs, instead of just one, thereby reducing the traffic per flow and providing more flows for load balancing. Second, our Enhanced-ECMP (E-ECMP) protocol effectively balances these 16 flows across different network paths by hashing on additional fields in the RoCE header of packets.
    
    \item \textbf{Congestion control.} We use deep-buffer switches in the spine~\citep{gangidi2024rmda} to accommodate transient congestion and buffering caused by collective communication patterns. This setup helps limit the impact of persistent congestion and network back pressure caused by slow servers, which is common in  training. Finally, better load balancing through E-ECMP significantly reduces the chance of congestion. With these optimizations, we successfully run a 24K GPU cluster without traditional congestion control methods such as Data Center Quantized Congestion Notification (DCQCN). 
\end{itemize}


\subsubsection{Parallelism for Model Scaling}
\label{section:4D-parallelism}

To scale training for our largest models, we use 4D parallelism—a combination of four different types of parallelism methods—to shard the model. This approach efficiently distributes computation across many GPUs and ensures each GPU's model parameters, optimizer states, gradients, and activations fit in its HBM. Our implementation of 4D parallelism is illustrated in Figure~\ref{fig:4d_parallelism}. It combines tensor parallelism (TP; \citet{NIPS2012_c399862d, shoeybi2019megatron, korthikanti2023reducing}), pipeline parallelism (PP; \citet{huang2019gpipe, narayanan2021efficient, lamy2023breadth}), context parallelism (CP; \citet{liu2023ring}), and data parallelism (DP; \citet{rajbhandari2020zeromemoryoptimizationstraining, ren2021zerooffloaddemocratizingbillionscalemodel, zhao2023pytorch}).

Tensor parallelism splits individual weight tensors into multiple chunks on different devices. Pipeline parallelism partitions the model vertically into stages by layers, so that different devices can process in parallel different stages of the full model pipeline. Context parallelism divides the input context into segments, reducing memory bottleneck for very long sequence length inputs. We use fully sharded data parallelism \citep[FSDP;][]{rajbhandari2020zeromemoryoptimizationstraining, ren2021zerooffloaddemocratizingbillionscalemodel, zhao2023pytorch}, which shards the model, optimizer, and gradients while implementing data parallelism which processes data in parallel on multiple GPUs and synchronizes after each training step. Our use of FSDP for Llama 3 shards optimizer states and gradients, but for model shards we do not reshard after forward computation to avoid an extra \texttt{all-gather} communication during backward passes.

\textbf{GPU utilization.}
Through careful tuning of the parallelism configuration, hardware, and software, we achieve an overall BF16 Model FLOPs Utilization (MFU; \citet{chowdhery2023palm}) of 38-43\% for the configurations shown in Table~\ref{table:mfu}.  The slight drop in MFU to 41\% on 16K GPUs with DP=128 compared to 43\% on 8K GPUs with DP=64 is due to the lower batch size per DP group needed to keep the global tokens per batch constant during training.

\begin{table}
	\centering
	\begin{tabular}{cccccccc|cc}
	\toprule
	     \textbf{GPUs} & \textbf{TP} & \textbf{CP} & \textbf{PP} & \textbf{DP}   & \textbf{Seq. Len.} &   \textbf{Batch size/DP} & \textbf{Tokens/Batch} & \textbf{TFLOPs/GPU} & \textbf{BF16 MFU}\\ 
	\midrule
	8,192    & 8 & 1 & 16 & 64   & 8,192   &   32 & 16M  & 430      & 43\%        \\
	16,384   & 8 & 1 & 16 & 128   & 8,192   &   16 & 16M  & 400      & 41\%        \\
	16,384   & 8 & 16 & 16 & 4   & 131,072 &   16 & 16M   & 380     & 38\%        \\
	\bottomrule

	\end{tabular}
\caption{\textbf{Scaling configurations and MFU for each stage of \llamathree 405B pre-training.} See text and Figure \ref{fig:4d_parallelism} for descriptions of each type of parallelism.}
\label{table:mfu}
\end{table}

\begin{figure}[t]
     \centering
     \includegraphics[width=\textwidth]{assets/4D_parallelism.pdf}
     \caption{\textbf{Illustration of 4D parallelism.} GPUs are divided into parallelism groups in the order of [TP, CP, PP, DP], where DP stands for FSDP. In this example, 16 GPUs are configured with a group size of |TP|=2, |CP|=2, |PP|=2, and |DP|=2. 
     A GPU's position in 4D parallelism is represented as a vector, [$D_1$, $D_2$, $D_3$, $D_4$], where $D_i$ is the index on the $i$-th parallelism dimension. In this example,
     GPU0[TP0, CP0, PP0, DP0] and GPU1[TP1, CP0, PP0, DP0] are in the same TP group, GPU0 and GPU2 are in the same CP group, GPU0 and GPU4 are in the same PP group, and GPU0 and GPU8 are in the same DP group.
     }
     \label{fig:4d_parallelism}
\end{figure}

\textbf{Pipeline parallelism improvements.}
We encountered several challenges with existing implementations:

\begin{itemize}
    \item \textbf{Batch size constraint.} Current implementations have constraints on supported batch size per GPU, requiring it to be divisible by the number of pipeline stages. For the example in Figure~\ref{fig:pipeline_parallelism}, the depth-first schedule (DFS) of pipeline parallelism~\citep{narayanan2021efficient} requires $N=\textrm{PP}=4$, while the breadth-first schedule (BFS; \citet{lamy2023breadth}) requires $N=M$, where $M$ is the total number of micro-batches and $N$ is the number of contiguous micro-batches for the same stage's forward or backward. However, pre-training often needs flexibility to adjust batch size.
    
    \item \textbf{Memory imbalance.} Existing pipeline parallelism implementations lead to imbalanced resource consumption. The first stage consumes more memory due to the embedding and the warm-up micro-batches.
    
    \item \textbf{Computation imbalance.} After the last layer of the model, we need to calculate output and loss, making this stage the execution latency bottleneck.
\end{itemize} 

To address these issues, we modify our pipeline schedule as shown in Figure~\ref{fig:pipeline_parallelism}, which allows setting $N$ flexibly---in this case $N=5$, which can run a arbitrary number of micro-batches in each batch. This allows us to run: (1) fewer micro-batches than the number of stages when we have batch size limit at large scale; or (2) more micro-batches to hide point-to-point communication, finding a sweet spot between DFS and breadth first schedule (BFS) for the best communication and memory efficiency. To balance the pipeline, we reduce one Transformer layer each from the first and the last stages, respectively. This means that the first model chunk on the first stage has only the embedding, and the last model chunk on the last stage has only output projection and loss calculation. To reduce pipeline bubbles, we use an interleaved schedule \citep{narayanan2021efficient} with $V$ pipeline stages on one pipeline rank. Overall pipeline bubble ratio is $\frac{\textrm{PP} - 1}{V * M}$. Further, we adopt asynchronous point-to-point communication in PP, which considerably speeds up training, especially in cases when the document mask introduces extra computation imbalance. We enable {\small \texttt{TORCH\_NCCL\_AVOID\_RECORD\_STREAMS}} to reduce memory usage from asynchronous point-to-point communication. Finally, to reduce memory cost, based on detailed memory allocation profiling, we proactively deallocate tensors that will not be used for future computation, including the input and output tensors of each pipeline stage, that will not be used for future computation. With these optimizations, we could pre-train \llamathree on sequences of 8K tokens without activation checkpointing.

\begin{figure*}[t]
     \centering
     \includegraphics[width=\textwidth]{assets/pipeline_parallelism.pdf}
     \caption{\textbf{Illustration of pipeline parallelism in Llama 3.} Pipeline parallelism partitions eight pipeline stages (0 to 7) across four pipeline ranks (PP ranks 0 to 3), where the GPUs with rank 0 run stages 0 and 4, the GPUs with P rank 1 run stages 1 and 5, \emph{etc}. The colored blocks (0 to 9) represent a sequence of micro-batches, where $M$ is the total number of micro-batches and $N$ is the number of continuous micro-batches for the same stage's forward or backward. Our key insight is to make $N$ tunable.
     }
     \label{fig:pipeline_parallelism}
\end{figure*}

\textbf{Context parallelism for long sequences.} We utilize context parallelism (CP) to improve memory efficiency when scaling the context length of \llamathree and enable training on extremely long sequences up to 128K in length. In CP, we partition across the sequence dimension, and specifically we partition the input sequence into $2 \times \mbox{CP}$ chunks so each CP rank receives two chunks for better load balancing. The $i$-th CP rank received both the $i$-th and the $(2 \times \mbox{CP} - 1 - i)$-th chunks. 

Different from existing CP implementations that overlap communication and computation in a ring-like structure~\citep{liu2023ring}, our CP implementation adopts an \texttt{all-gather} based method where we first \texttt{all-gather} the key (K) and value (V) tensors, and then compute attention output for the local query (Q) tensor chunk. Although the \texttt{all-gather} communication latency is exposed in the critical path, we still adopt this approach for two main reasons: (1) it is easier and more flexible to support different types of attention masks in \texttt{all-gather} based CP attention, such as the document mask; and (2) the exposed \texttt{all-gather} latency is small as the communicated K and V tensors are much smaller than Q tensor due to the use of GQA \citep{ainslie2023gqa}. Hence, the time complexity of attention computation is an order of magnitude larger than \texttt{all-gather} ($O(S^2)$ versus $O(S)$, where $S$ represents the sequence length in the full causal mask), making the \texttt{all-gather} overhead negligible.

\textbf{Network-aware parallelism configuration.} The order of parallelism dimensions, [TP, CP, PP, DP], is optimized for network communication. The innermost parallelism requires the highest network bandwidth and lowest latency, and hence is usually constrained to within the same server. The outermost parallelism may spread across a multi-hop network and should tolerate higher network latency. Therefore, based on the requirements for network bandwidth and latency, we place parallelism dimensions in the order of [TP, CP, PP, DP]. DP (\emph{i.e.}, FSDP) is the outermost parallelism because it can tolerate longer network latency by asynchronously prefetching sharded model weights and reducing gradients. Identifying the optimal parallelism configuration with minimal communication overhead while avoiding GPU memory overflow is challenging. We develop a memory consumption estimator and a performance-projection tool which helped us explore various parallelism configurations and project overall training performance and identify memory gaps effectively.

\textbf{Numerical stability.} By comparing training loss between different parallelism setups, we fixed several numerical issues that impact training stability. To ensure training convergence, we use FP32 gradient accumulation during backward computation over multiple micro-batches and also \texttt{reduce-scatter} gradients in FP32 across data parallel workers in FSDP. For intermediate tensors, \emph{e.g.}, vision encoder outputs, that are used multiple times in the forward computation, the backward gradients are also accumulated in FP32.

\subsubsection{Collective Communication}
\label{sec:ncclx}

Our collective communication library for \llamathree is based on a fork of Nvidia's NCCL library, called NCCLX. NCCLX significantly improves the performance of NCCL, especially for higher latency networks. Recall that the order of parallelism dimensions is [TP, CP, PP, DP], where DP corresponds to FSDP. The outermost parallelism dimensions, PP and DP, may communicate through a multi-hop network, with latency up to tens of microseconds. The original NCCL collectives---\texttt{all-gather} and \texttt{reduce-scatter} in FSDP, and \texttt{point-to-point} in PP---require data chunking and staged data copy. This approach incurs several inefficiencies, including (1) requiring a large number of small control messages to be exchanged over the network to facilitate data transfer, (2) extra memory-copy operations, and (3) using extra GPU cycles for communication.  For \llamathree training, we address a subset of these inefficiencies by tuning chunking and data transfer to fit our network latencies, which can be as high as tens of microseconds for a large cluster. We also allow small control messages to traverse our network at a higher priority, especially avoiding being head-of-line blocked in deep-buffer core switches. Our ongoing work for future Llama versions involves making deeper changes in NCCLX to holistically address all the aforementioned problems.

\subsubsection{Reliability and Operational Challenges}

The complexity and potential failure scenarios of 16K GPU training surpass those of much larger CPU clusters that we have operated. Moreover, the synchronous nature of training makes it less fault-tolerant---a single GPU failure may require a restart of the entire job. Despite these challenges, for \llamathree, we achieved higher than 90\% effective training time while supporting automated cluster maintenance, such as firmware and Linux kernel upgrades~\citep{leonhardi2024maintenance}, which resulted in at least one~training interruption daily. The effective training time measures the time spent on useful training over the elapsed time.

During a 54-day snapshot period of pre-training, we experienced a total of 466 job interruptions. Of these, 47 were planned interruptions due to automated maintenance operations such as firmware upgrades or operator-initiated operations like configuration or dataset updates. The remaining 419 were unexpected interruptions, which are classified in Table~\ref{table:job_interruptions}.
Approximately 78\% of the unexpected interruptions are attributed to confirmed hardware issues, such as GPU or host component failures, or suspected hardware-related issues like silent data corruption and unplanned individual host maintenance events. GPU issues are the largest category, accounting for 58.7\% of all unexpected issues.  Despite the large number of failures, significant manual intervention was required only three times during this period, with the rest of issues handled by automation. 

\begin{table}[]
\centering
\begin{tabular}{lccc}
    \toprule
\textbf{Component}             & \textbf{Category} & \textbf{Interruption Count} & \textbf{\% of Interruptions} \\
\midrule
Faulty GPU                            & GPU               & 148                         & 30.1\%                       \\
GPU HBM3 Memory                & GPU               & 72                          & 17.2\%                       \\
Software Bug                   & Dependency          & 54                          & 12.9\%                       \\
Network Switch/Cable           & Network           & 35                          & 8.4\%                        \\
Host Maintenance               & \begin{tabular}[c]{@{}c@{}}Unplanned \\ Maintenance\end{tabular}       & 32                          & 7.6\%                        \\
GPU SRAM Memory                & GPU               & 19                          & 4.5\%                        \\
GPU System Processor           & GPU               & 17                          & 4.1\%                        \\
NIC                            & Host              & 7                           & 1.7\%                        \\
NCCL Watchdog Timeouts                  & Unknown           & 7                           & 1.7\%                        \\
Silent Data Corruption                      & GPU           & 6                           & 1.4\%                        \\
GPU Thermal Interface + Sensor & GPU               & 6                           & 1.4\%                        \\
SSD                            & Host              & 3                           & 0.7\%                        \\
Power Supply                   & Host              & 3                           & 0.7\%                        \\
Server Chassis                 & Host              & 2                           & 0.5\%                        \\
IO Expansion Board             & Host              & 2                           & 0.5\%                        \\
Dependency                     & Dependency        & 2                           & 0.5\%                        \\
CPU                            & Host              & 2                           & 0.5\%                        \\
System Memory                  & Host              & 2                           & 0.5\%    \\
\bottomrule
\end{tabular}
\caption{\textbf{Root-cause categorization of unexpected interruptions during a 54-day period of Llama 3 405B pre-training.} About 78\% of unexpected interruptions were attributed to confirmed or suspected hardware issues. }
\label{table:job_interruptions}
\end{table}

To increase the effective training time, we reduced job startup and checkpointing time, and developed tools for fast diagnosis and problem resolution. We extensively use PyTorch's built-in NCCL flight recorder~\citep{ansel2024pytorch}, a feature that captures collective metadata and stack traces into a ring buffer, and hence allowing us to diagnose hangs and performance issues quickly at scale, particularly with regard to NCCLX. Using this, we efficiently record every communication event and the duration of each collective operation, and also automatically dump tracing data on NCCLX watchdog or heartbeat timeout. We enable more computationally intensive tracing operations and metadata collection selectively as needed live in production through online configuration changes~\citep{configerator} without needing a code release or job restart.

Debugging issues in large-scale training is complicated by the mixed use of NVLink and RoCE in our network. Data transfer over NVLink typically occurs through load/store operations issued by CUDA kernels, and failures in either the remote GPU or NVLink connectivity often manifest as stalled load/store operations within CUDA kernels without returning a clear error code. NCCLX enhances the speed and accuracy of failure detection and localization through a tight co-design with PyTorch, allowing PyTorch to access NCCLX’s internal state and track relevant information. While stalls due to NVLink failures cannot be completely prevented, our system monitors the state of the communication library and automatically times out when such a stall is detected. Additionally, NCCLX traces the kernel and network activities of each NCCLX communication and provides a snapshot of the failing NCCLX collective's internal state, including finished and pending data transfers between all ranks. We analyze this data to debug NCCLX scaling issues.

Sometimes, hardware issues may cause still-functioning but slow stragglers that are hard to detect. Even a single straggler can slow down thousands of other GPUs, often appearing as functioning but slow communications. We developed tools to prioritize potentially problematic communications from selected process groups. By investigating just a few top suspects,  we were usually able to effectively identify the stragglers.

One interesting observation is the impact of environmental factors on training performance at scale. For Llama 3 405B , we noted a diurnal 1-2\% throughput variation based on time-of-day. This fluctuation is the result of higher mid-day temperatures impacting GPU dynamic voltage and frequency scaling.

During training, tens of thousands of GPUs may increase or decrease power consumption at the same time, for example, due to all GPUs waiting for checkpointing or collective communications to finish, or the startup or shutdown of the entire training job. When this happens, it can result in instant fluctuations of power consumption across the data center on the order of tens of megawatts, stretching the limits of the power grid. This is an ongoing challenge for us as we scale training for future, even larger Llama models.

\subsubsection{Speech Understanding}

Training of the speech module is done in two stages.
The first stage, speech pre-training, leverages unlabeled data to train a speech encoder that exhibits strong generalization capabilities across languages and acoustic conditions.
In the second stage, supervised fine-tuning, the adapter and pre-trained encoder are integrated with the language model, and trained jointly with it while the LLM stays frozen. This enables the model to respond to speech input.
This stage uses labeled data corresponding to speech understanding abilities.

Multilingual ASR and AST modeling often results in language confusion/interference, which leads to degraded performance. A popular way to mitigate this is to incorporate language identification (LID) information, both on the source and target side. This can lead to improved performance in the predetermined set of directions, but it does come with potential loss of generality. For instance, if a translation system expects LID on both source and target side, then the model will not likely to show good zero-shot performance in directions that were not seen in training.
So our challenge is to design a system that allows LID information to some extent, but keeps the model general enough such that we can have the model do speech translation in unseen directions.
To address this, we design system prompts which only contain LID for the text to be emitted (target side). There is no LID information for the speech input (source side) in these prompts, which also potentially allows it to work with code-switched speech.
For ASR, we use the following system prompt: {\tt Repeat after me in \{language\}:},
where {\tt \{language\}} comes from one of the 34 languages (English, French, \emph{etc.})
For speech translation, the system prompt is: {\tt Translate the following sentence into \{language\}:}.
This design has been shown to be effective in prompting the language model to respond in the desired language.
We used the same system prompts during training and inference.

\textbf{Speech pre-training.}
We use the self-supervised BEST-RQ algorithm \citep{chiu2022self} to pre-train the speech encoder.
We apply a mask of 32-frame length with a probability of 2.5\% to the input mel-spectrogram.
If the speech utterances are longer than 60 seconds, we perform a random crop of 6K frames, corresponding to 60 seconds of speech.
We quantize mel-spectrogram features by stacking 4 consecutive frames, projecting the 320-dimensional vectors to a 16-dimensional space, and performing a nearest-neighbor search with respect to cosine similarity metric within a codebook of 8,192 vectors.
To stabilize pre-training, we employ 16 different codebooks.
The projection matrix and codebooks are randomly initialized and are not updated throughout the model training.
The multi-softmax loss is used only on masked frames for efficiency reasons.
The encoder is trained for 500K steps with a global batch size of 2,048 utterances.

\textbf{Supervised finetuning.}
Both the pre-trained speech encoder and the randomly initialized adapter are further jointly optimized with \llamathree in the supervised finetuning stage. The language model remains unchanged during this process.
The training data is a mixture of ASR, AST, and spoken dialogue data.
The speech model for Llama 3 8B is trained for 650K updates, using a global batch size of 512 utterances and an initial learning rate of $10^{-4}$.
The speech model for Llama 3 70B is trained for 600K updates, using a global batch size of 768 utterances and an initial learning rate of $4\times10^{-5}$.

\subsection{Post-Training}
\label{section:vision_post_training}
In this section, we describe the post-training recipe for our vision adapters. After pre-training, we fine-tune the model on highly curated multi-modal conversational data to enable chat capabilities. We further implement direct preference optimization (DPO) to boost human evaluation performance and rejection sampling to improve multi-modal reasoning capabilities. Finally, we add a quality-tuning stage where we continue fine-tuning the model on a very small set of high-quality conversational data which further boosts human evaluation while retaining performance across benchmarks. More details on each of these steps are provided below.

\subsubsection{Supervised Finetuning Data}
\label{subsubsection:vision_supervised_finetuning_data}
We describe our supervised finetuning (SFT) data for image and video capabilities separately below.

\textbf{Image.} We utilize a mix of different datasets for supervised finetuning.
\begin{itemize}
\item \textbf{Academic datasets.} We convert a highly filtered collection of existing academic datasets to question-answer pairs using templates or via LLM rewriting. The LLM rewriting's purpose is to augment the data with different instructions and to improve the language quality of answers.

\item \textbf{Human annotations.} We collect multi-modal conversation data via human annotators for a wide range of tasks (open-ended question-answering, captioning, practical use cases, \textit{etc.}) and domains (\textit{e.g.}, natural images and structured images). Annotators are provided with images and asked to write conversations. To ensure diversity, we cluster large-scale datasets and sampled images uniformly across different clusters. Further, we acquire additional images for a few specific domains by expanding a seed via k-nearest neighbors. Annotators are also provided with intermediate checkpoints of existing models to facilitate model-in-the-loop style annotations, so that model generations can be utilized as a starting point by the annotators to then provide additional human edits. This is an iterative process, in which model checkpoints would be regularly updated with better performing versions trained on the latest data. This increases the volume and efficiency of human annotations, while also improving their quality.

\item \textbf{Synthetic data.} We explore different ways to generate synthetic multi-modal data by using text-representations of images and a text-input LLM. The high-level idea is to utilize the reasoning capabilities of text-input LLMs to generate question-answer pairs in the text domain, and replace the text representation with its corresponding images to produce synthetic multi-modal data. Examples include rendering texts from question-answer datasets as images or rendering table data into synthetic images of tables and charts. Additionally, we use captions and OCR extractions from existing images to generate additional conversational or question-answer data related to the images.
\end{itemize}

\textbf{Video.} Similar to the image adapter, we use academic datasets with pre-existing annotations and convert them into appropriate textual instructions and target responses.
The targets are converted to open-ended responses or multiple-choice options, whichever is more appropriate.
We ask humans to annotate videos with questions and corresponding answers.
The annotators are asked to focus on questions that could not be answered based on a single frame, to steer the annotators towards questions that require temporal understanding.

\subsubsection{Supervised Finetuning Recipe}
We describe our supervised finetuning (SFT) recipe for image and video capabilities separately below.

\label{subsubsection:vision_supervised_finetuning_recipe}
\textbf{Image.} We initialize from the pre-trained image adapter, but hot-swap the pre-trained language model's weights with the instruction tuned language model's weights. The language model weights are kept frozen to maintain text-only performance, \textit{i.e.}, we only update the vision encoder and image adapter weights.

Our approach to finetune the model is similar to \cite{wortsman2022modelsoupsaveragingweights}. First, we run a hyperparameter sweep using multiple random subsets of data, learning rates and weight decay values. Next, we rank the models based on their performance. Finally, we average the weights of the top-$K$ models to obtain the final model. The value of $K$ is determined by evaluating the averaged models and selecting the instance with highest performance. We observe that the averaged models consistently yield better results compared to the best individual model found via grid search. Further, this strategy reduces sensitivity to hyperparameters.

\textbf{Video.} For video SFT, we initialize the video aggregator and cross-attention layers using the pre-trained weights.
The rest of the parameters in the model, the image weights and the LLM, are initialized from corresponding models following their finetuning stages.
Similar to video pre-training, we then finetune only the video parameters on the video SFT data.
For this stage, we increase the video length to 64 frames, and use an aggregation factor of 32 to get two effective frames.
The resolution of the chunks is also increased to be consistent with the corresponding image hyperparameters.

\subsubsection{Preference Data}
\label{subsubsection:vision_preference_data}
We built multimodal pair-wise preference datasets for reward modeling and direct preference optimization.
\begin{itemize}

\item \textbf{Human annotations.} The human-annotated preference data consists of comparisons between two different model outputs, labeled as ``chosen'' and ``rejected'', with 7-scale ratings. The models used to generate responses are sampled on-the-fly from a pool of the best recent models, each with different characteristics. We update the model pool weekly. Besides preference labels, we also request annotators to provide optional human edits to correct inaccuracies in ``chosen'' responses because vision tasks have a low tolerance for inaccuracies. Note that human editing is an optional step because there is a trade-off between volume and quality in practice.

\item \textbf{Synthetic data.} Synthetic preference pairs could also be generated by using text-only LLMs to edit and deliberately introduce errors in the supervised finetuning dataset. We took the conversational data as input, and use an LLM to introduce subtle but meaningful errors (\textit{e.g.}, change objects, change attributes, add mistakes in calculations, etc.). These edited responses are used as negative ``rejected'' samples and paired with the ``chosen'' original supervised finetuning data.

\item \textbf{Rejection sampling.} Furthermore, to create more \emph{on-policy} negative samples, we leveraged the iterative process of rejection sampling to collect additional preference data. We discuss our usage of rejection sampling in more detail in the following sections. At a high-level, rejection sampling is used to iteratively sample high-quality generations from a model. Therefore, as a by-product, all generations that are not selected can be used as negative rejected samples and used as additional preference data pairs.

\end{itemize}

\subsubsection{Reward Modeling}
\label{subsubsection:vision_reward_modeling}
We train a vision reward model (RM) on top of the vision SFT model and the language RM. The vision encoder and the cross-attention layers are initialized from the vision SFT model and unfrozen during training, while the self-attention layers are initialized from the language RM and kept frozen. We observe that freezing the language RM part generally leads to better accuracy, especially on tasks that require the RM to judge based on its knowledge or the language quality. We adopt the same training objective as the language RM, but adding a weighted regularization term on the square of the reward logits averaged over the batch, which prevents the reward scores from drifting.

The human preference annotations in Section~\ref{subsubsection:vision_preference_data} are used to train the vision RM. We follow the same practice as language preference data (Section~\ref{sec:rlhf_annotation_data}) to create two or three pairs with clear ranking (\emph{edited} > \emph{chosen} > \emph{rejected}). In addition, we also synthetically augment the negative responses by perturbing the words or phrases related to the information in the image (such as numbers or visual texts). This encourages the vision RM to ground its judgement based on the actual image content.

\subsubsection{Direct Preference Optimization}
\label{subsubsec:dpo}
Similar to the language model (Section~\ref{subsubsec:postdpo}), we further train the vision adapters with Direct Preference Optimization (DPO;~\cite{rafailov2023dpo}) using the preference data described in Section~\ref{subsubsection:vision_preference_data}. To combat the distribution shift during post-training rounds, we only keep recent batches of human preference annotations while dropping batches that are sufficiently off-policy (\textit{e.g.}, if the base pre-trained model is changed). We find that instead of always freezing the reference model, updating it in an exponential moving average (EMA) fashion every k-steps helps the model learn more from the data, resulting in better performance in human evaluations. Overall, we observed that the vision DPO model consistently performs better than its SFT starting point in human evaluations for every finetuning iteration.

\subsubsection{Rejection Sampling}
\label{subsubsection:vision_rejection_sampling}
Most available question-answer pairs only contain the final answer and lack the chain-of-thought explanation that is required to train a model that generalizes well for reasoning tasks.
We use rejection sampling to generate the missing explanations for such examples and boost the model's reasoning capabilities.

Given a question-answer pair, we generate multiple answers by sampling the finetuned model with different system prompts or temperature.
Next, we compare the generated answers to the ground-truth via heuristics or an LLM judge.
Finally, we retrain the model by adding the correct answers back into the finetuning data mix. We find it useful to keep multiple correct answers per question.

To ensure we only add high-quality examples back into training, we implemented the following two guardrails.
First, we find that some examples contain incorrect explanations, despite the final answer being correct.
We observed that this pattern occurs more frequently for questions where only a small fraction of the generated answers is correct.
Therefore, we drop answers for questions where the probability of the answer being correct is below a certain threshold.
Second, raters prefer some answers over others due to differences in language or style.
We use the reward model to select top-$K$ highest-quality answers and add them back into training.

\subsubsection{Quality Tuning}
\label{subsubsection:vision_quality_tuning}
We curate a small but \emph{highly} selective SFT dataset where all samples have been rewritten and verified either by humans or our best models to meet our highest standards. We train DPO models with this data to improve response quality, calling the process Quality-Tuning (QT). We find that QT significantly improves human evaluations without affecting generalization verified by benchmarks when the QT dataset covers a wide range of tasks and proper early stopping is applied. We select checkpoints at this stage purely based on benchmarks to ensure capabilities are retained or improved.

\subsection{Image Recognition Results}
\label{section:results_image_recognition}

We evaluate the performance of the image understanding capabilities of \llamathree on a range of tasks spanning natural image understanding, text understanding, charts understanding and multimodal reasoning:
\begin{itemize}
\item \textbf{MMMU}~\citep{yue2023mmmu} is a challenging dataset for mulitmodal reasoning where model is expected to understand
images and solve college-level problems spanning 30 different disciplines. This includes both multiple-choice and open ended
questions. We evaluate our model on the validation set with 900 images, in line with other works.

\item \textbf{VQAv2}~\citep{vqav2} tests the ability of a model to combine image understanding, language understanding and
commonsense knowlege to answer generic questions about natural images

\item \textbf{AI2 Diagram}~\citep{Kembhavi2016ADI} evaluates models capability to parse scientific diagrams
and answer questions about the same. We use the same evaluation protocol as Gemini and x.ai, and report scores using a transparent bounding box.

\item \textbf{ChartQA}~\citep{masry-etal-2022-chartqa} is a challenging benchmark for charts understanding. This requires
model to visually understand different kinds of charts and answer logical questions about the charts.

\item \textbf{TextVQA}~\citep{singh2019towards} is a popular benchmark dataset that requires
models to read and reason about text in images to answer questions about them. This tests the
OCR understanding ability of the model on natural images.

\item \textbf{DocVQA}~\citep{Mathew2020DocVQAAD} is a benchmark dataset focused on document analysis and recognition.
It contains images of a wide range of documents which evaluates a model's ability to perform OCR understanding
and reason about the contents of a document to answer questions about them.
\end{itemize}

Table~\ref{table:image_recognition} presents the results of our experiments.
The results in the table show that our vision module attached to \llamathree performs competitively across a wide range of image-recognition benchmarks at varying model capacities.
Using the resulting \llamathree-V 405B model, we outperform GPT-4V on all benchmarks, while being slightly behind Gemini 1.5 Pro and Claude 3.5 Sonnet.
\llamathree 405B appears particularly competitive on document understanding tasks.

\begin{table}[t]
    \centering
    \resizebox{\linewidth}{!}{\subsection{Image Recognition Results}
\label{section:results_image_recognition}

We evaluate the performance of the image understanding capabilities of \llamathree on a range of tasks spanning natural image understanding, text understanding, charts understanding and multimodal reasoning:
\begin{itemize}
\item \textbf{MMMU}~\citep{yue2023mmmu} is a challenging dataset for mulitmodal reasoning where model is expected to understand
images and solve college-level problems spanning 30 different disciplines. This includes both multiple-choice and open ended
questions. We evaluate our model on the validation set with 900 images, in line with other works.

\item \textbf{VQAv2}~\citep{vqav2} tests the ability of a model to combine image understanding, language understanding and
commonsense knowlege to answer generic questions about natural images

\item \textbf{AI2 Diagram}~\citep{Kembhavi2016ADI} evaluates models capability to parse scientific diagrams
and answer questions about the same. We use the same evaluation protocol as Gemini and x.ai, and report scores using a transparent bounding box.

\item \textbf{ChartQA}~\citep{masry-etal-2022-chartqa} is a challenging benchmark for charts understanding. This requires
model to visually understand different kinds of charts and answer logical questions about the charts.

\item \textbf{TextVQA}~\citep{singh2019towards} is a popular benchmark dataset that requires
models to read and reason about text in images to answer questions about them. This tests the
OCR understanding ability of the model on natural images.

\item \textbf{DocVQA}~\citep{Mathew2020DocVQAAD} is a benchmark dataset focused on document analysis and recognition.
It contains images of a wide range of documents which evaluates a model's ability to perform OCR understanding
and reason about the contents of a document to answer questions about them.
\end{itemize}

Table~\ref{table:image_recognition} presents the results of our experiments.
The results in the table show that our vision module attached to \llamathree performs competitively across a wide range of image-recognition benchmarks at varying model capacities.
Using the resulting \llamathree-V 405B model, we outperform GPT-4V on all benchmarks, while being slightly behind Gemini 1.5 Pro and Claude 3.5 Sonnet.
\llamathree 405B appears particularly competitive on document understanding tasks.

\begin{table}[t]
    \centering
    \resizebox{\linewidth}{!}{\subsection{Image Recognition Results}
\label{section:results_image_recognition}

We evaluate the performance of the image understanding capabilities of \llamathree on a range of tasks spanning natural image understanding, text understanding, charts understanding and multimodal reasoning:
\begin{itemize}
\item \textbf{MMMU}~\citep{yue2023mmmu} is a challenging dataset for mulitmodal reasoning where model is expected to understand
images and solve college-level problems spanning 30 different disciplines. This includes both multiple-choice and open ended
questions. We evaluate our model on the validation set with 900 images, in line with other works.

\item \textbf{VQAv2}~\citep{vqav2} tests the ability of a model to combine image understanding, language understanding and
commonsense knowlege to answer generic questions about natural images

\item \textbf{AI2 Diagram}~\citep{Kembhavi2016ADI} evaluates models capability to parse scientific diagrams
and answer questions about the same. We use the same evaluation protocol as Gemini and x.ai, and report scores using a transparent bounding box.

\item \textbf{ChartQA}~\citep{masry-etal-2022-chartqa} is a challenging benchmark for charts understanding. This requires
model to visually understand different kinds of charts and answer logical questions about the charts.

\item \textbf{TextVQA}~\citep{singh2019towards} is a popular benchmark dataset that requires
models to read and reason about text in images to answer questions about them. This tests the
OCR understanding ability of the model on natural images.

\item \textbf{DocVQA}~\citep{Mathew2020DocVQAAD} is a benchmark dataset focused on document analysis and recognition.
It contains images of a wide range of documents which evaluates a model's ability to perform OCR understanding
and reason about the contents of a document to answer questions about them.
\end{itemize}

Table~\ref{table:image_recognition} presents the results of our experiments.
The results in the table show that our vision module attached to \llamathree performs competitively across a wide range of image-recognition benchmarks at varying model capacities.
Using the resulting \llamathree-V 405B model, we outperform GPT-4V on all benchmarks, while being slightly behind Gemini 1.5 Pro and Claude 3.5 Sonnet.
\llamathree 405B appears particularly competitive on document understanding tasks.

\begin{table}[t]
    \centering
    \resizebox{\linewidth}{!}{\input{results/tables/image_recognition}}
    \caption{\textbf{Image understanding performance of our vision module attached to \llamathree.} We compare model performance to GPT-4V, GPT-4o, Gemini 1.5 Pro, and Claude 3.5 Sonnet. $^{\triangle}$Results obtained using external OCR tools.}
    \label{table:image_recognition}
\end{table}
}
    \caption{\textbf{Image understanding performance of our vision module attached to \llamathree.} We compare model performance to GPT-4V, GPT-4o, Gemini 1.5 Pro, and Claude 3.5 Sonnet. $^{\triangle}$Results obtained using external OCR tools.}
    \label{table:image_recognition}
\end{table}
}
    \caption{\textbf{Image understanding performance of our vision module attached to \llamathree.} We compare model performance to GPT-4V, GPT-4o, Gemini 1.5 Pro, and Claude 3.5 Sonnet. $^{\triangle}$Results obtained using external OCR tools.}
    \label{table:image_recognition}
\end{table}

\subsection{Video Recognition Results}
\label{section:results_video_recognition}

We evaluate our video adapter for Llama 3 on three benchmarks:
\begin{itemize}
\item \textbf{PerceptionTest}~\citep{patraucean2023perception} evaluates the model's ability to answer temporal reasoning questions focusing on skills (memory, abstraction, physics, semantics) and different types of reasoning (descriptive, explanatory, predictive, counterfactual). It consists of $11.6K$ test QA pairs, each with an on-average $23s$ long video, filmed by $100$ participants worldwide to show perceptually interesting tasks. We focus on the multiple-choice question answering task, where each question is paired with three possible options. We report performance on the held-out test split which is accessed by submitting our predictions to an online challenge server.\footnote{See \url{https://eval.ai/web/challenges/challenge-page/2091/overview}.}

\item \textbf{NExT-QA}~\citep{xiao2021next} is another temporal and causal reasoning benchmark, with a focus on open-ended question answering.
It consists of $1K$ test videos each on-average $44s$ in length, paired with $9K$ questions. The evaluation is performed by comparing the model's responses with the ground truth answer using Wu-Palmer Similarity (WUPS)~\citep{wu1994verb}.\footnote{See \url{https://github.com/doc-doc/NExT-OE}.}

\item \textbf{TVQA}~\citep{lei2018tvqa} evaluates the model's ability to perform compositional reasoning, requiring spatiotemporal localization of relevant moments, recognition of visual concepts, and joint reasoning with subtitle-based dialogue. This dataset, being derived from popular TV shows, additionally tests for the model's ability to leverage its outside-knowledge of those TV shows in answering the questions. It consists of over $15K$ validation QA pairs, with each corresponding video clip being on-average $76s$ in length. It also follows a multiple-choice format with five options for each question, and we report performance on the validation set following prior work~\citep{openai2023gpt4blog}.

\item \textbf{ActivityNet-QA}~\citep{yu2019activityqa} evaluates the model's ability to reason over long video clips to understand actions, spatial relations, temporal relations, counting, etc. It consists of $8K$ test QA pairs from $800$ videos, each on-average $3$ minutes long. For evaluation, we follow the protocol from prior work~\citep{gemini2023gemini,lin2023video,Maaz2023VideoChatGPT}, where the model generates short one-word or one-phrase answers, and the correctness of the output is evaluated using the GPT-3.5 API which compares it to the ground truth answer. We report the average accuracy as evaluated by the API.
\end{itemize}

When performing inference, we uniformly sample frames from the full video clip and pass those frames into the model with a short text prompt. Since most of our benchmarks involve answering multiple-choice questions, we use the following prompt: {\tt Select the correct answer from the following options: \{question\}.  Answer with the correct option letter and nothing else}. For benchmarks that require producing a short answer ({\em e.g.}, ActivityNet-QA and NExT-QA), we use the following prompt: {\tt Answer the question using a single word or phrase. \{question\}}. For NExT-QA, since the evaluation metric (WUPS) is sensitive to the length and the specific words used, we additionally prompt the model to be specific and respond with the most salient answer, for instance specifying ``living room'' instead of simply responding with ``house'' when asked a location question. For benchmarks that contain subtitles ({\em i.e.}, TVQA), we include the subtitles corresponding to the clip in the prompt during inference.

\begin{table}[t]
    \centering
    \resizebox{\linewidth}{!}{\subsection{Video Recognition Results}
\label{section:results_video_recognition}

We evaluate our video adapter for Llama 3 on three benchmarks:
\begin{itemize}
\item \textbf{PerceptionTest}~\citep{patraucean2023perception} evaluates the model's ability to answer temporal reasoning questions focusing on skills (memory, abstraction, physics, semantics) and different types of reasoning (descriptive, explanatory, predictive, counterfactual). It consists of $11.6K$ test QA pairs, each with an on-average $23s$ long video, filmed by $100$ participants worldwide to show perceptually interesting tasks. We focus on the multiple-choice question answering task, where each question is paired with three possible options. We report performance on the held-out test split which is accessed by submitting our predictions to an online challenge server.\footnote{See \url{https://eval.ai/web/challenges/challenge-page/2091/overview}.}

\item \textbf{NExT-QA}~\citep{xiao2021next} is another temporal and causal reasoning benchmark, with a focus on open-ended question answering.
It consists of $1K$ test videos each on-average $44s$ in length, paired with $9K$ questions. The evaluation is performed by comparing the model's responses with the ground truth answer using Wu-Palmer Similarity (WUPS)~\citep{wu1994verb}.\footnote{See \url{https://github.com/doc-doc/NExT-OE}.}

\item \textbf{TVQA}~\citep{lei2018tvqa} evaluates the model's ability to perform compositional reasoning, requiring spatiotemporal localization of relevant moments, recognition of visual concepts, and joint reasoning with subtitle-based dialogue. This dataset, being derived from popular TV shows, additionally tests for the model's ability to leverage its outside-knowledge of those TV shows in answering the questions. It consists of over $15K$ validation QA pairs, with each corresponding video clip being on-average $76s$ in length. It also follows a multiple-choice format with five options for each question, and we report performance on the validation set following prior work~\citep{openai2023gpt4blog}.

\item \textbf{ActivityNet-QA}~\citep{yu2019activityqa} evaluates the model's ability to reason over long video clips to understand actions, spatial relations, temporal relations, counting, etc. It consists of $8K$ test QA pairs from $800$ videos, each on-average $3$ minutes long. For evaluation, we follow the protocol from prior work~\citep{gemini2023gemini,lin2023video,Maaz2023VideoChatGPT}, where the model generates short one-word or one-phrase answers, and the correctness of the output is evaluated using the GPT-3.5 API which compares it to the ground truth answer. We report the average accuracy as evaluated by the API.
\end{itemize}

When performing inference, we uniformly sample frames from the full video clip and pass those frames into the model with a short text prompt. Since most of our benchmarks involve answering multiple-choice questions, we use the following prompt: {\tt Select the correct answer from the following options: \{question\}.  Answer with the correct option letter and nothing else}. For benchmarks that require producing a short answer ({\em e.g.}, ActivityNet-QA and NExT-QA), we use the following prompt: {\tt Answer the question using a single word or phrase. \{question\}}. For NExT-QA, since the evaluation metric (WUPS) is sensitive to the length and the specific words used, we additionally prompt the model to be specific and respond with the most salient answer, for instance specifying ``living room'' instead of simply responding with ``house'' when asked a location question. For benchmarks that contain subtitles ({\em i.e.}, TVQA), we include the subtitles corresponding to the clip in the prompt during inference.

\begin{table}[t]
    \centering
    \resizebox{\linewidth}{!}{\subsection{Video Recognition Results}
\label{section:results_video_recognition}

We evaluate our video adapter for Llama 3 on three benchmarks:
\begin{itemize}
\item \textbf{PerceptionTest}~\citep{patraucean2023perception} evaluates the model's ability to answer temporal reasoning questions focusing on skills (memory, abstraction, physics, semantics) and different types of reasoning (descriptive, explanatory, predictive, counterfactual). It consists of $11.6K$ test QA pairs, each with an on-average $23s$ long video, filmed by $100$ participants worldwide to show perceptually interesting tasks. We focus on the multiple-choice question answering task, where each question is paired with three possible options. We report performance on the held-out test split which is accessed by submitting our predictions to an online challenge server.\footnote{See \url{https://eval.ai/web/challenges/challenge-page/2091/overview}.}

\item \textbf{NExT-QA}~\citep{xiao2021next} is another temporal and causal reasoning benchmark, with a focus on open-ended question answering.
It consists of $1K$ test videos each on-average $44s$ in length, paired with $9K$ questions. The evaluation is performed by comparing the model's responses with the ground truth answer using Wu-Palmer Similarity (WUPS)~\citep{wu1994verb}.\footnote{See \url{https://github.com/doc-doc/NExT-OE}.}

\item \textbf{TVQA}~\citep{lei2018tvqa} evaluates the model's ability to perform compositional reasoning, requiring spatiotemporal localization of relevant moments, recognition of visual concepts, and joint reasoning with subtitle-based dialogue. This dataset, being derived from popular TV shows, additionally tests for the model's ability to leverage its outside-knowledge of those TV shows in answering the questions. It consists of over $15K$ validation QA pairs, with each corresponding video clip being on-average $76s$ in length. It also follows a multiple-choice format with five options for each question, and we report performance on the validation set following prior work~\citep{openai2023gpt4blog}.

\item \textbf{ActivityNet-QA}~\citep{yu2019activityqa} evaluates the model's ability to reason over long video clips to understand actions, spatial relations, temporal relations, counting, etc. It consists of $8K$ test QA pairs from $800$ videos, each on-average $3$ minutes long. For evaluation, we follow the protocol from prior work~\citep{gemini2023gemini,lin2023video,Maaz2023VideoChatGPT}, where the model generates short one-word or one-phrase answers, and the correctness of the output is evaluated using the GPT-3.5 API which compares it to the ground truth answer. We report the average accuracy as evaluated by the API.
\end{itemize}

When performing inference, we uniformly sample frames from the full video clip and pass those frames into the model with a short text prompt. Since most of our benchmarks involve answering multiple-choice questions, we use the following prompt: {\tt Select the correct answer from the following options: \{question\}.  Answer with the correct option letter and nothing else}. For benchmarks that require producing a short answer ({\em e.g.}, ActivityNet-QA and NExT-QA), we use the following prompt: {\tt Answer the question using a single word or phrase. \{question\}}. For NExT-QA, since the evaluation metric (WUPS) is sensitive to the length and the specific words used, we additionally prompt the model to be specific and respond with the most salient answer, for instance specifying ``living room'' instead of simply responding with ``house'' when asked a location question. For benchmarks that contain subtitles ({\em i.e.}, TVQA), we include the subtitles corresponding to the clip in the prompt during inference.

\begin{table}[t]
    \centering
    \resizebox{\linewidth}{!}{\input{results/tables/video_recognition}}
    \caption{\textbf{Video understanding performance of our vision module attached to Llama 3.} We find that across range of tasks covering long-form and temporal video understanding, our vision adapters for \llama{3} 8B and 70B parameters are competitive and sometimes even outperform alternative models.}
    \label{table:video_recognition}
\end{table}

We present the performance of Llama 3 8B and 70B in Table~\ref{table:video_recognition}.
We compare Llama 3's performance with that of two Gemini and two GPT-4 models. Note that all our results are zero-shot, as we do not include any part of these benchmarks in our training or finetuning data. We find that our Llama 3 models that train a small video adapter during post-training are very competitive, and in some cases even better, than other models that potentially leverage native multimodal processing all the way from pre-training.
Llama 3 performs particularly well on video recognition given that we only evaluate the 8B and 70B parameter models.
Llama 3 achieves its best performance on PerceptionTest, suggesting the model has a strong ability to perform complex temporal reasoning.  On long-form activity understanding tasks like ActivityNet-QA, Llama 3 is able to obtain strong results even though it is processing only up to 64 frames, which means that for a 3-minute long video the model only processes one frame every 3 seconds.
}
    \caption{\textbf{Video understanding performance of our vision module attached to Llama 3.} We find that across range of tasks covering long-form and temporal video understanding, our vision adapters for \llama{3} 8B and 70B parameters are competitive and sometimes even outperform alternative models.}
    \label{table:video_recognition}
\end{table}

We present the performance of Llama 3 8B and 70B in Table~\ref{table:video_recognition}.
We compare Llama 3's performance with that of two Gemini and two GPT-4 models. Note that all our results are zero-shot, as we do not include any part of these benchmarks in our training or finetuning data. We find that our Llama 3 models that train a small video adapter during post-training are very competitive, and in some cases even better, than other models that potentially leverage native multimodal processing all the way from pre-training.
Llama 3 performs particularly well on video recognition given that we only evaluate the 8B and 70B parameter models.
Llama 3 achieves its best performance on PerceptionTest, suggesting the model has a strong ability to perform complex temporal reasoning.  On long-form activity understanding tasks like ActivityNet-QA, Llama 3 is able to obtain strong results even though it is processing only up to 64 frames, which means that for a 3-minute long video the model only processes one frame every 3 seconds.
}
    \caption{\textbf{Video understanding performance of our vision module attached to Llama 3.} We find that across range of tasks covering long-form and temporal video understanding, our vision adapters for \llama{3} 8B and 70B parameters are competitive and sometimes even outperform alternative models.}
    \label{table:video_recognition}
\end{table}

We present the performance of Llama 3 8B and 70B in Table~\ref{table:video_recognition}.
We compare Llama 3's performance with that of two Gemini and two GPT-4 models. Note that all our results are zero-shot, as we do not include any part of these benchmarks in our training or finetuning data. We find that our Llama 3 models that train a small video adapter during post-training are very competitive, and in some cases even better, than other models that potentially leverage native multimodal processing all the way from pre-training.
Llama 3 performs particularly well on video recognition given that we only evaluate the 8B and 70B parameter models.
Llama 3 achieves its best performance on PerceptionTest, suggesting the model has a strong ability to perform complex temporal reasoning.  On long-form activity understanding tasks like ActivityNet-QA, Llama 3 is able to obtain strong results even though it is processing only up to 64 frames, which means that for a 3-minute long video the model only processes one frame every 3 seconds.

