\begin{figure}
  
\footnotesize
\begin{framed}

You are the coding expert and your task is to evaluate the difficulty of a given coding prompt. \\

Here is the definition of three difficulty level (easy, medium, hard) for coding prompts: \\

\#\# Easy

The prompt is undergraduate level questions or leetcode easy question. The prompt requires limited domain /algorithmics knowledge or implementation context (architecture, libraries, pre-existing code). There is little ambiguity in the prompt (in case of underspecification, good default behaviors are easy to come up with or not important), limited complexity of specifications (in number of instructions). The solution of prompt is easy to explain (e..g, code doesn’t need comments to be understood) and to test for/debug (e.g., limited corner cases). \\

For example:

\begin{itemize}
    \item Implement the factorial function in <language X>.
    \item How to count the number of lines in a file called log.txt in bash?
    \item How can I parse a json file in <language X/library Y>?
    \item What’s the complexity of getting the n-th element from a list in Python?
\end{itemize} 

\vspace{5mm}

\#\# Medium 

The prompt is masters level questions or leetcode medium question. The prompt requires knowledge of standard algorithms and data structures to get an optimal solution, knowledge of common libraries and concepts. May require additional code context. There is Medium ambiguity in the prompt (e.g., needs to come up with reasonable ad-hoc data representation or class structure without explicit guidance), multiple requirements should be satisfied or multiple bugs should be found. \\

For example:

\begin{itemize}
    \item Compute the average value in the column .metrics.accuracy for elements with .compile=True in the file metrics.jsonl in bash
    \item Compute the longest increasing subsequence in a list. Implement the solution in <language X>. 
    \item How can I use the twitter python API to send a tweet automatically?
\end{itemize}

\vspace{5mm}
\#\# Hard

The prompt is domain expert level or leetcode hard question. The prompt requires understanding of complex and long code/log snippet). The prompt requires expert domain knowledge, or information on the specific application or deployment scenario, including substantial specific API/code context. Finding good solutions of the prompt needs non-trivial design decisions regarding data structures, algorithms or code architecture/design patterns. Finding a solution requires solving several non-trivial subproblems or finding non-trivial bugs. Problem involves tricky corner cases, explaining the solution to a non-expert requires adding context. \\

For example:

\begin{itemize}
    \item Example 1 (leetcode hard type):
For my homework I need to write a python function that implements regular expression matching, with support for '.' and '+', where '.' stands for any character and '+' matches 1 or more of the preceding element. The function has signature match(s, p) where s is the input string and p the regexp pattern. The matching may be partial. I can't use any regexp library for this assignment. Could you help with that?
\item Example 2 (expert knowledge):
What would be AdagradW, the equivalent of AdamW for Adagrad? Implement AdagradW in Python.

\item Example 3: (objective-oriented design question) Make a pong game in <language X>

\item Example 4 (complex plot):
I have a pandas dataframe with the columns "decoding", "Capabilities", "Fine-tuning", "Model size", "HE pass@1", "MBPP pass@1". I want a seaborn figure with two scatterplots side-by-side. The two plots show "HE pass@1" vs "MBPP pass@1", using different subsets of the data: The first plot uses the data with "decoding" equal to 0.1, the second plot uses
"greedy" for "decoding".

\item Example 5 (documentation/comments/unit tests/commits for complex and long code/log snippet)

\item Example 6 (create example usages/code Summarization of complex and long code snippet):
\end{itemize}

Based on the definition of the three difficulty levels, please evaluate the difficulty of the last user prompt. You need to provide your explanation first. Then, output the level of difficulty **strictly following this XML format**: <difficulty></difficulty>. For example, <difficulty>Easy</difficulty>

\end{framed}

  \caption{\textbf{Llama code difficulty scoring prompt}: We use the above prompt to score the difficulty of synthetically generated coding data.}
  \label{fig:llama_difficulty_coding_prompt}
\end{figure}



