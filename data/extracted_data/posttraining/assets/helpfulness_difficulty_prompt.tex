\begin{figure}

\small
\begin{framed}

Our task is to evaluate the difficulty of a given user prompt. \\
Here is the definition of three difficulty levels (Easy, Medium, Hard) for user prompts:\\


\#\# Easy

Prompt is a single ask/requirement/constraint for the model presented as a single statement OR prompt is a single statement without ask/requirement/constraints.
Prompt would not require subject matter expertise to understand. \\

Examples:
\begin{itemize}
\item Illustrate and explain the proper use of a semi-colon.
\item How do I uninvite my brother to my wedding?
\item I've been having trouble sticking to my healthy diet lately. Give me some motivational words or tips to help me make better food choices and achieve my health goals.
\end{itemize}

\vspace{5mm}

\#\# Medium

Prompt includes 2-4 asks/requirements/constraints for the model AND would not require subject matter expertise to produce a response. \\

Examples:
\begin{itemize}
\item My neighbors blast loud music all night, and I can’t sleep. I’ve tried talking to them directly, as well as calling 311 but nothing has changed. What else do you think I can try?
\item How do I ask my boss for a raise? I think I’m underpaid but my boss never has time for me.
\item Pretend you’re Bugs Bunny. I’m Elmer Fudd. How would you greet me?
\item Write me a funny haiku about dogs.
\end{itemize}

\vspace{5mm}

\#\# Hard

Prompt contains 5 or more asks/requirements/constraints for the model OR requires subject matter expertise above and beyond “common knowledge” in order to respond. \\

Examples:
\begin{itemize}
\item Write a poem to say sorry to my dog because I didn't spend enough time with it. The poem should have 26 lines where each line begins with Z, Y, X, ..., A, respectively, and always ends with h. The poem cannot contain any animal words.
\item Sort the following words alphabetically, and in the result remove the first and the fourth words while capitalize the rest: sioux fortescue purloin percept helmsman friend friends. Append a new lower-case word that is an animal living in Antarctica. Output the result with numbered bullets.
\item Handling long-sequence inputs presents a significant challenge to the KV-cache of Transformers.  Can we address this challenge better by training Transformers with more GPUs?
\end{itemize}

Based on the definition of the three difficulty levels, please evaluate the difficulty of the last user prompt. You need to provide your explanation first. Then, output the level of difficulty strictly following this XML format: <difficulty></difficulty>. For example, <difficulty>Easy</difficulty>

\end{framed}

\caption{\textbf{Llama English difficulty scoring prompt}: We use the above prompt to score the difficulty of general English data.}
\label{fig:llama_difficulty_helpful_prompt}
\end{figure}
