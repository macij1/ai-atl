\section{Speech Experiments}
\label{section:speech}

We perform experiments to study a compositional approach of integrating speech capabilities into Llama 3, resembling the method we used for visual recognition. On the input side, an encoder, together with an adapter, is incorporated to process speech signals. We leverage a system prompt (in text) to enable different modes of operation for speech understanding in Llama 3.
If no system prompt is provided, the model acts as a general-purpose spoken dialogue model which can effectively respond to the user speech in a manner that is consistent with the text-only version of Llama 3.
The dialogue history is introduced as the prompt prefix to improve the multi-round dialogue experience.
We also experiment with system prompts that enable the use of Llama 3 for automatic speech recognition (ASR) and automatic speech translation (AST).
The speech interface of Llama 3 supports up to 34 languages.\footnote{The speech interface supports the following 34 languages:
	Arabic,
	Bengali,
	Chinese,
	Czech,
	Dutch,
	English,
	Finnish,
	French,
	German,
	Greek,
	Gujarati,
	Hindi,
	Hungarian,
	Indonesian,
	Italian,
	Japanese,
	Kannada,
	Korean,
	Malayalam,
	Marathi,
	Persian,
	Polish,
	Portuguese,
	Romanian,
	Russian,
	Spanish,
	Swahili,
	Swedish,
	Tamil,
	Telugu,
	Thai,
	Turkish,
	Urdu,
	Vietnamese.}
It also allows for the interleaved input of text and speech, enabling the model to solve advanced audio-comprehension tasks.

We also experiment with a speech generation approach in which we implement a streaming text-to-speech (TTS) system that generates speech waveforms on-the-fly during language model decoding. We design the speech generator for Llama 3 based on a proprietary TTS system and do not fine-tune the language model for speech generation. Instead, we focus on improving speech synthesis latency, accuracy, and naturalness by leveraging Llama 3 embeddings at inference time.
The speech interface is illustrated in Figure~\ref{sph:fig:multimodal_model_overview} and~\ref{sph:fig:model}.

\begin{figure}
    \centering
    \includegraphics[width=\textwidth]{assets/llama-speech-6x.png}
    \caption{\textbf{Architecture of our speech interface for Llama 3.}}
    \label{sph:fig:model}
\end{figure}

\subsubsection{Speech Generation}
\label{sec:data:tts}
The speech generation datasets mainly consist of those for training the text normalization (TN) model and the prosody model (PM).  Both training data are augmented with an additional input feature of the Llama 3 embeddings to provide contextual information.

\textbf{Text normalization data.} Our TN training dataset includes 55K samples that cover a wide range of semiotic classes (\emph{e.g.}, number, date, time) that require non-trivial normalization. Each sample is a pair of written-form text and the corresponding normalized spoken-form text, with an inferred sequence of handcrafted TN rules that carry out the normalization.

\textbf{Prosody model data.} The PM training data includes linguistic and prosodic features extracted from a 50K-hour TTS dataset, which are paired transcripts and audios recorded by professional voice actors in studio settings.

\textbf{Llama 3 embedding.}
The Llama 3 embeddings are taken as the output of the 16th decoder layer. We work exclusively with the Llama 3 8B model and extract the embeddings for a given text (\emph{i.e.} written-form input text for TN or the audio transcript for PM) as if they are generated by the Llama 3 model with an empty user prompt. In a given sample, each chunk in the Llama 3 token sequence is explicitly aligned with the corresponding chunks in native input sequence for TN or PM, \emph{i.e.}, TN-specific text tokens (demarcated by unicode category) or phone-rate features respectively. This allows for training the TN and PM modules with streaming input of Llama 3 tokens and embeddings.

\subsection{Model Architecture}
\label{section:pretraining_model_architecture}

Llama 3 uses a standard, dense Transformer architecture~\citep{vaswani2017attention}.
It does not deviate significantly from Llama and Llama 2 \citep{touvron2023llama,touvron2023llama2} in terms of model architecture; our performance gains are primarily driven by improvements in data quality and diversity as well as by increased training scale.

We make a few small modifications compared to Llama 2:
\begin{itemize}
    \item We use grouped query attention (GQA; \citet{ainslie2023gqa}) with 8 key-value heads to improve inference speed and to reduce the size of key-value caches during decoding.
    \item We use an attention mask that prevents self-attention between different documents within the same sequence.
    We find that this change had limited impact during in standard pre-training, but find it to be important in continued pre-training on very long sequences.
    \item We use a vocabulary with 128K tokens. Our token vocabulary combines 100K tokens from the \texttt{tiktoken}\footnote{\url{https://github.com/openai/tiktoken/tree/main}} tokenizer with 28K additional tokens to better support non-English languages. Compared to the Llama 2 tokenizer, our new tokenizer improves compression rates on a sample of English data from 3.17 to 3.94 characters per token. This enables the model to ``read'' more text for the same amount of training compute. We also found that adding 28K tokens from select non-English languages improved both compression ratios and downstream performance, with no impact on English tokenization.
    \item We increase the RoPE base frequency hyperparameter to 500,000. This enables us to better support longer contexts; \citet{xiong2023effective} showed this value to be effective for context lengths up to 32,768.
\end{itemize}

\begin{table}[]
	\centering
	\begin{tabular}{l|ccc}
	\toprule
	                      & \textbf{8B}  & \textbf{70B}  & \textbf{405B}\\
	\midrule
	Layers       & 32           & 80            & 126         \\
	Model Dimension   & 4,096         & 8192          & 16,384       \\
	FFN Dimension        &     14,336         &      28,672         &    53,248      \\
	Attention Heads    & 32           & 64            & 128         \\
	Key/Value Heads       & 8            & 8             & 8           \\
	Peak Learning Rate    & $3 \times 10^{-4}$         & $1.5  \times 10^{-4}$       & $8 \times 10^{-5}$        \\
	Activation Function   & \multicolumn{3}{c}{SwiGLU}                 \\
	Vocabulary Size       & \multicolumn{3}{c}{128,000}                   \\
	Positional Embeddings & \multicolumn{3}{c}{RoPE ($\theta=500,000$)} \\
	\bottomrule
	\end{tabular}
	\caption{\textbf{Overview of the key hyperparameters of Llama 3.} We display settings for 8B, 70B, and 405B language models.}
	\label{table:overview_model_hyperparams}
\end{table}

Llama 3 405B uses an architecture with 126 layers, a token representation dimension of 16,384, and 128 attention heads; see Table~\ref{table:overview_model_hyperparams} for details.
This leads to a model size that is approximately compute-optimal according to scaling laws on our data for our training budget of $3.8 \times 10^{25}$ FLOPs.

\subsubsection{Scaling Laws}
\label{section:scaling_law}

We develop scaling laws \citep{hoffmann2022chinchilla,kaplan2020scaling} to determine the optimal model size for our flagship model given our pre-training compute budget.
In addition to determining the optimal model size, a major challenge is to forecast the flagship model's performance on downstream benchmark tasks, due to a couple of issues: (1) Existing scaling laws typically predict only next-token prediction loss rather than specific benchmark performance.
(2)  Scaling laws can be noisy and unreliable because they are developed based on pre-training runs conducted with small compute budgets~\citep{wei2022emergent}.

To address these challenges, we implement a two-stage methodology to develop scaling laws that accurately predict downstream benchmark performance:
\begin{enumerate}
    \item We first establish a correlation between the compute-optimal model's negative log-likelihood on downstream tasks and the training FLOPs.
    \item Next, we correlate the negative log-likelihood on downstream tasks with task accuracy, utilizing both the scaling law models and older models trained with higher compute FLOPs. In this step, we specifically leverage the Llama 2 family of models.
\end{enumerate}
This approach enables us to predict downstream task performance given a specific number of training FLOPs for compute-optimal models.
We use a similar method to select our pre-training data mix (see Section~\ref{section:pretraining_training_recipe}).


\textbf{Scaling law experiments.}
Concretely, we construct our scaling laws by pre-training models using compute budgets between $6 \times 10^{18}$ FLOPs and $10^{22}$ FLOPs.
At each compute budget, we pre-train models ranging in size between 40M and 16B parameters, using a subset of model sizes at each compute budget.
In these training runs, we use a cosine learning rate schedule with a linear warmup for 2,000 training steps.
The peak learning rate is set between $2 \times 10^{-4}$ and $4 \times 10^{-4}$ depending on the size of the model.
We set the cosine decay to 0.1 of the peak value.
The weight decay at each step is set to 0.1 times the learning rate at that step.
We use a fixed batch size for each compute scale, ranging between 250K and 4M.


\begin{figure}[tbp]
	\centering
	\begin{minipage}{0.45\textwidth}
		\centering
		\includegraphics[width=0.9\textwidth]{assets/isoflops.pdf}
		\caption{\textbf{Scaling law IsoFLOPs curves} between $6 \times 10^{18}$ and $10^{22}$ FLOPs. The loss is the negative log-likelihood on a held-out validation set. We approximate measurements at each compute scale using a second degree polynomial.}
		\label{fig:scaling_law_isoflops}
	\end{minipage}\hfill%
	\begin{minipage}{0.45\textwidth}
		\centering
		\includegraphics[width=0.9\textwidth]{assets/datacompute.pdf}
		\caption{\textbf{Number of training tokens in identified compute-optimal models as a function of pre-training compute budget.} We include the fitted scaling-law prediction as well. The compute-optimal models correspond to the parabola minimums in Figure~\ref{fig:scaling_law_isoflops}.}
		\label{fig:data_compute_scaling_law_fit}
	\end{minipage}
\end{figure}

These experiments give rise to the IsoFLOPs curves in Figure~\ref{fig:scaling_law_isoflops}.
The loss in these curves is measured on a separate validation set.
We fit the measured loss values using a second-degree polynomial and identify the minimums of each parabola.
We refer to minimum of a parabola as the \emph{compute-optimal} model at the corresponding pre-training compute budget.


We use the compute-optimal models we identified this way to predict the optimal number of training tokens for a specific compute budget.
To do so, we assume a power-law relation between compute budget, $C$, and the optimal number of training tokens, $N^\star(C)$:
\begin{align*}
    N^\star(C) = A C^\alpha.
\end{align*}
We fit $A$ and $\alpha$ using the data from Figure~\ref{fig:scaling_law_isoflops}.
We find that $(\alpha, A) = (0.53, 0.29)$; the corresponding fit is shown in Figure~\ref{fig:data_compute_scaling_law_fit}.
Extrapolation of the resulting scaling law to $3.8 \times 10^{25}$ FLOPs suggests training a
402B parameter model on 16.55T tokens.


An important observation is that IsoFLOPs curves become \emph{flatter} around the minimum as the compute budget increases.
This implies that performance of the flagship model is relatively robust to small changes in the trade-off between model size and training tokens.
Based on this observation, we ultimately decided to train a flagship model with 405B parameters.


\textbf{Predicting performance on downstream tasks.} We use the resulting compute-optimal models to forecast the performance of the flagship Llama 3 model on benchmark data sets. 
First, we linearly correlate the (normalized) negative log-likelihood of correct answer in the benchmark and the training FLOPs. In this analysis, we use only the scaling law models trained up to $10^{22}$ FLOPs on the data mix described above.
Next, we establish a sigmoidal relation between the log-likelihood and accuracy using both the scaling law models and Llama 2 models, which were trained using the Llama 2 data mix and tokenizer.
We show the results of this experiment on the ARC Challenge benchmark in Figure~\ref{fig:scaling_law_benchmarks}).
We find this two-step scaling law prediction, which extrapolates over four orders of magnitude, to be quite accurate: it only slightly underestimates the final performance of the flagship Llama 3 model.

\begin{figure}[tbp]
	\centering
	\includegraphics[width=0.7\textwidth]{assets/scaling_laws_benchmark.pdf}
	\caption{\textbf{Scaling law forecast for ARC Challenge.} \emph{Left:} Normalized negative log-likelihood of the correct answer on the ARC Challenge benchmark as a function of pre-training FLOPs.
	\emph{Right:} ARC Challenge benchmark accuracy as a function of the normalized negative log-likelihood of the correct answer. This analysis enables us to predict model performance on the ARC Challenge benchmark before pre-training commences. See text for details.}
	\label{fig:scaling_law_benchmarks}
\end{figure}

\subsection{Pre-training}
\label{section:vision_training_recipe}

\textbf{Image.}
We initialize from the pre-trained text model and vision encoder weights.
The vision encoder is unfrozen, while the text model weights are kept frozen as explained above.
First, we train the model using 6B image-text pairs where each image is resized to fit within four tiles of $336 \times 336$ pixels.
We use a global batch size of 16,384 and a cosine learning rate schedule with initial learning rate $10 \times 10^{-4}$ and a weight decay of $0.01$.
The initial learning rate was determined based on small-scale experiments.
However,  these findings did not generalize well to very long training schedules and dropped the learning rate a few times during training when the loss values became stagnant.
After the base pre-training, we increase the image resolution further and continue training the same weights on the annealing dataset.
The optimizer is re-initialized via warm-up to learning rate $2 \times 10^{-5}$ and again follows a cosine schedule.

\textbf{Video.}
For video pre-training, we start from the image pre-trained and annealed weights as described above.
We add the video aggregator and cross-attention layers as described in the architecture, initialized randomly. We freeze all the parameters in the model except the video-specific ones (the aggregator and video cross-attention), and train them on the video pre-training data.
We use the same training hyperparameters as the image annealing stage, with small differences in the learning rate.
We uniformly sample 16 frames from the full video, and represent each frame using four chunks, each of size of $448 \times 448$ pixels.
We use an aggregation factor of 16 in the video aggregator, hence obtaining one effective frame, which the text tokens cross-attend to.
We use a global batch size of 4,096, a sequence length of 190 tokens, and a learning rate of $10^{-4}$ during training.



\subsection{Speech Understanding Results}
\label{section:results_speech}
We evaluate the speech understanding capabilities of our speech interface for Llama 3 on three tasks: \textbf{(1)} automatic speech recognition, \textbf{(2)} speech translation, and \textbf{(3)} spoken question answering.
We compare the performance of our speech interface for Llama 3 with three state-of-the-art models for speech understanding: Whisper \citep{radford23whisper}, SeamlessM4T \citep{barrault2023seamless}, and Gemini.\footnote{
	Due to technical limitations, we compare with the performance of Gemini on MLS reported in the original paper.
}
In all the evaluations, we used greedy search for Llama 3 token prediction.

\textbf{Speech recognition.}
We evaluate the ASR performance on the
English datasets of
Multilingual LibriSpeech (MLS; \citet{pratap2020mls}),
LibriSpeech \citep{panayotov2015librispeech},
VoxPopuli \citep{wang2021voxpopuli},
and a subset of the multilingual FLEURS dataset \citep{conneau2023fleurs}.
In evaluation, the decoding results are post-processed using the Whisper text normalizer to ensure consistency in comparing with the reported results of other models.
On all benchmarks, we measure the word error rate of our speech interface for Llama 3 on the standard test set of those benchmarks, except for Chinese, Japanese, Korean and Thai, where the character error rate is reported.



\providecommand{\bup}{($\boldsymbol\uparrow$)}
\providecommand{\bdown}{($\boldsymbol\downarrow$)}


\begin{table}[t]
	\centering
	 \resizebox{\linewidth}{!}{\begin{NiceTabular}{lcccccc}
	\CodeBefore
	\Body
	\toprule
	& \textbf{Llama 3 8B} & \textbf{Llama 3 70B} & \textbf{Whisper} & \textbf{SeamlessM4T v2} & \textbf{Gemini 1.0 Ultra} & \textbf{Gemini 1.5 Pro}\\
	\midrule
	MLS \scriptsize{(English)} & 4.9 & 4.4 & 6.2 \scriptsize{(v2)} & 6.5 & 4.4 & \textbf{4.2} \\
	LibriSpeech \scriptsize{(test-other)} & 3.4 & \textbf{3.1} & 4.9 \scriptsize{(v2)} & 6.2 & -- &  -- \\
	VoxPopuli \scriptsize{(English)}  & 6.2 & \textbf{5.7} &  7.0  \scriptsize{(v2)} & 7.0 & -- & --  \\
	FLEURS \scriptsize{(34 languages)} & 9.6 & \textbf{8.2} & 14.4 \scriptsize{(v3)}  & 11.7 & -- & -- \\
	\bottomrule
\end{NiceTabular}
}
	\caption{\textbf{Word error rate of our speech interface for Llama 3 on speech recognition tasks.} We report the performance of Whisper, SeamlessM4T, and Gemini for reference.}
	\label{table:speech_asr_results}
\end{table}


Table~\ref{table:speech_asr_results} shows the results of ASR evaluations.
It demonstrates the strong performance of Llama 3 (and multi-modal foundation models more generally) on speech recognition tasks: our model outperforms models that are tailored to speech like Whisper\footnote{On FLEURS ASR, Malayalam is not officially reported for Whisper v3, so we use the average of 33 languages.} and SeamlessM4T on all benchmarks.
On MLS English, Llama 3 performs similarly to Gemini.


\begin{table}[t]
	\centering
	\begin{NiceTabular}{lcccc}
	\CodeBefore
	\Body
	\toprule
	& \textbf{Llama 3 8B} & \textbf{Llama 3 70B} & \textbf{Whisper v2} & \textbf{SeamlessM4T v2}\\
	\midrule
	FLEURS \scriptsize{(33 lang. $\rightarrow$ English)} & 29.5 & \textbf{33.7}  & 21.9  & 28.6  \\
	Covost 2 \scriptsize{(15 lang. $\rightarrow$ English)} & 34.4 & \textbf{38.8} & 33.8  & 37.9 \\
	\bottomrule
\end{NiceTabular}

	\caption{\textbf{BLEU score of our speech interface for Llama 3 on speech translation tasks.} We report the performance of Whisper and SeamlessM4T for reference.}
	\label{table:speech_ast_results}
\end{table}


\textbf{Speech translation.}
We also evaluate our models on speech translation tasks in which the model is asked to translate non-English speech into English text.
We use the FLEURS and Covost 2 \citep{wang2021covost} datasets in these evaluations, measuring BLEU scores of the translated English.
Table~\ref{table:speech_ast_results} presents the results of these experiments.\footnote{On Covost 2, we evaluate only on 15 (out of 21) languages.} 
The performance of our models in speech translation highlights the advantages of multimodal foundation models for tasks such as speech translation. 

\begin{figure}[]
    \centering
    \includegraphics[trim={150px 0 150px 0},clip,width=\textwidth]{assets/llama-voice-language.pdf}
    \caption{\textbf{Transcribed dialogue examples using the speech interface for Llama 3.} The examples illustrate zero-shot multi-turn and code-switching capabilities.}
    \label{figure:speech_dialog_example}
\end{figure}

\textbf{Spoken question answering.}
The speech interface of Llama 3 demonstrates remarkable question answering capabilities. The model can effortlessly comprehend code-switched speech without any prior exposure to such data. Notably, although the model was trained only on single-turn dialogue, it is capable of engaging in extended, coherent multi-turn dialogue sessions.
Figure~\ref{figure:speech_dialog_example} presents a few examples that highlight these multilingual and multi-turn capabilities.

\begin{table}[t]
\centering
    \begin{tabular}{lcccccc}
    \toprule
     & \multicolumn{2}{c}{\textbf{Llama 3 8B}} &  \multicolumn{2}{c}{\textbf{Llama 3 70B}} & \multicolumn{2}{c}{\textbf{Gemini 1.5 Pro}} \\
    \textbf{Language} & AT \bdown & LT \bup & AT \bdown & LT  \bup & AT \bdown & LT \bup \\
    \midrule
    English & 0.84 & 15.09 & \textbf{0.68} & \textbf{15.46} & 1.44 & 13.42 \\
    Overall & 2.31 & 9.89 & \textbf{2.00} & 10.29 & 2.06 & \textbf{10.94} \\
    \bottomrule
    \end{tabular}
    \caption{\textbf{Speech toxicity of our speech interface to Llama 3 on the MuTox dataset.} AT refers to added toxicity (\%) and LT refers to lost toxicity (\%). \label{table:speech-safety-mutox}}
\end{table}

\textbf{Safety.}
We evaluate the safety of our speech model on MuTox \citep{mutox}, a multilingual audio-based dataset of 20,000 utterances for English and Spanish and 4,000 for 19 other languages, each with toxicity labels attached.
The audio is passed as input to the model and the output is evaluated for toxicity, after cleaning some special characters.
We apply the MuTox classifier~\citep{mutox} and compare the results with Gemini 1.5 Pro. We evaluate the percentage of added toxicity (AT), when the input prompt is safe and the output is toxic, and the percentage of lost toxicity (LT), when the input prompt is toxic and the answer is safe. Table~\ref{table:speech-safety-mutox} shows the results for English and an average across all 21 languages that we evaluated on.\footnote{Note that for Gemini, we encountered that a significant number of responses were empty, which could be due to safety filters on their side (though some empty responses were for non-toxic input) or to rate limits. To conduct the analysis, we assumed that all the empty responses are safe. This is the most conservative approach for results and the upper bound of what Gemini results would look like.} The percentage of added toxicity is very low: our speech models have the lowest percentage of added toxicity for English, with less than 1\%. It removes significantly more toxicity than it adds.


\subsection{Speech Generation Results}
For speech generation, we focus on evaluating the quality of token-wise input streaming models with the Llama 3 embeddings for the text normalization and prosody modeling tasks. The evaluation focuses on comparisons with models that do not take the Llama 3 embeddings as an additional input.

\textbf{Text normalization.}
To measure the effect of Llama 3 embeddings, we experimented with changing the amount of right context the model uses. We trained the model using a right context of 3 TN tokens (demarcated by unicode category). This model is compared to models that do not use the Llama 3 embeddings, using a 3-token right context or a full bi-directional context.
As expected, Table~\ref{tab:table1} shows using the full right context improves performance for the model without Llama 3 embeddings. However, the model that incorporates the Llama 3 embeddings outperforms all other models, hence enabling token-rate input/output streaming without relying on long context in the input.

\begin{wraptable}{r}{0.45\textwidth}
	\begin{NiceTabular}{lcc}
		\CodeBefore
		\Body
		\toprule
		\textbf{Model} & \textbf{Context} & \textbf{Accuracy} \\
		\midrule
		Without Llama 3 8B & 3 & 73.6\% \\
		Without Llama 3 8B & $\infty$ & 88.0\% \\
		With Llama 3 8B & 3 & \textbf{90.7\%} \\
		\bottomrule
	\end{NiceTabular}
	\caption{\textbf{Sample-wise text normalization (TN) accuracy.} We compare models with or without Llama 3 8B embeddings, and using different right-context values.\vspace{-8mm}}
	\label{tab:table1}
\end{wraptable}

\textbf{Prosody modeling.}
To evaluate the performance of the our prosody model (PM) with Llama 3 8B, we conducted two sets of human evaluation comparing models with and without Llama 3 embeddings. Raters listened to samples from different models and indicated their preferences. To generate the final speech waveform, we use an in-house transformer based acoustic model \citep{wu2021transformer} that predicts spectral features and a WaveRNN neural vocoder \citep{kalchbrenner2018efficient} to generate the final speech waveform.  %

\begin{table}[t]
	\centering
    \begin{minipage}{.48\textwidth}
      \centering
      \begin{tabular}{lcc}
		\toprule
		\textbf{Model} & \textbf{Preference} \\
		\midrule
		PM for Llama 3 8B  & \textbf{60.0\%} \\
		\small{Streaming phone-only baseline} & 40.0\% \\
		\bottomrule
      \end{tabular}
		\label{tab:tts:pm:ab_test1}
    \end{minipage}\hfill
    \begin{minipage}{.48\textwidth}
      \centering
      \begin{tabular}{lcc}
		\toprule
		\textbf{Model} & \textbf{Preference} \\
		\midrule
		PM for Llama 3 8B & \textbf{63.6\%} \\
		\small{Non-streaming phone-only baseline} & 36.4\% \\
		\bottomrule
      \end{tabular}

    \end{minipage}
    \caption{\textbf{Prosody Modeling (PM) evaluation.} \emph{Left:} Rater preferences of PM for Llama 3 8B vs. streaming phone-only baseline. \emph{Right:} Rater preferences of PM for Llama 3 8B vs. non-streaming phone-only baseline.}
    \label{tab:pm_test}

\end{table}




First, we compare directly to a streaming baseline model without Llama 3 embeddings. 
In  the second test, the Llama 3 8B PM is compared to a non-streaming baseline model without Llama 3 embeddings. 
As shown in Table~\ref{tab:pm_test}, the Llama 3 8B PM is preferred  60\% of the time compared to the streaming baseline, and 63.6\% of the time  compared to the non-streaming baseline, indicating a significant improvement in perceived quality. The key advantage of the Llama 3 8B PM is its token-wise streaming capability (Section~\ref{sec:tts:pm}), which maintains low latency during inference. This reduces the model's lookahead requirements, enabling more responsive and real-time speech synthesis compared to non-streaming baselines.
Overall, the Llama 3 8B prosody model consistently outperforms the baseline models, demonstrating its effectiveness in enhancing the naturalness and expressiveness of synthesized speech.

