\subsection{Preliminaries}
Before introducing our learning algorithm, some auxiliary notions are
necessary to give a concrete representation of nominal languages
and automata.
%
In fact, binders yield infinitely many equivalent representation of nominal words
due to alpha-conversion.
%
For instance, $\ll \aname. \aname \gg$ and
$\ll \aname[m].\aname[m] \gg$ are the same nominal word \emph{up-to}
renaming of their bound name.
%
We introduce \emph{canonical expressions} to give a finitary
representation of nominal regular languages.

\begin{definition}[Canonical expressions]\label{def:CE}
  Let $1 \leq n  \in \Nat$ a natural number and $ne$ a closed nominal
  regular expression.
  %
  The \emph{$n$-canonical representation} $\ce{n}{ne}$ of $ne$ is
  defined as follows
  \begin{itemize}
  \item $ne \in \{\emptyword,\emptyset\}\cup \Sigma \implies \ce{n}{ne} =ne$
  \item $\ce{n}{ne+ne'} =\ce{n}{ne}+\ce{n}{ne'}$
  \item $\ce{n}{ne\cdot ne'} =\ce{n}{ne}\cdot \ce{n}{ne'}$,
  \item $\ce{n}{ne^*} =(\ce{n}{ne})^*$
  \item $ne=\langle \aname. ne'\rangle \implies \ce{n}{ne} =\langle n. \ce{n+1}{ne'[n/\aname]}\rangle$
  \end{itemize}
  where $ne'[n/\aname]$ is the capture-avoiding substitution of
  $\aname$ for $n$ in $ne'$.
  %
  The \emph{canonical representation} of $ne$ is the term
  $\ce 1 {ne}$.
\end{definition}
%
Note that the map $\ce{\_}{\_}$ does not change the structure of the
nominal regular expression $ne$.  Basically, $\ce \_ \_$ maps nominal
regular expressions to terms where names are concretely represented as
positive numbers.
\begin{example}
  Given $\Sigma=\{a,b\}$, we give some examples of canonical
  representations of nominal expressions.
  \begin{small}
  \begin{itemize}
  \item $aba$ is the canonical representations of itself; indeed $\ce{1}{aba} = aba$
  \item $\ce{1}{\langle \aname . a\, \aname \rangle} = \ce{1}{\langle \aname[m]. a\, \aname[m] \rangle}{}=\langle 1.a\,1 \rangle$
    is the canonical representation of both $\langle \aname.a\,\aname \rangle$ and $\langle \aname[m].a\,\aname[m] \rangle$
  \item the canonical representation of
    $ne = \langle \aname.a\,\aname \langle \aname[m].\aname\, b\, \aname[m]
    \rangle\rangle\langle \aname[m].\aname[m] \rangle$ is
    $\ce{1} {ne} = \langle 1.a1\ce{2}{(\langle \aname[m].1\, b \, \aname[m] \rangle)}\langle
    1.1\rangle \\=\langle 1.a\,1\langle 2.1\,b\,2 \rangle \rangle\langle 1.1
    \rangle$.
  \end{itemize}  
  \end{small}
  Note that the map $\ce{\_}{\_}$ replaces names with numbers so that
  alpha-equivalent expressions are mapped to the same term (second example above).
\end{example}
%
Canonical expressions are the linguistic counter part of the mechanism
used in the definition of $(\Sigma, \Nom_{fin})$-automaton give
in~\cite{Kurz0T13}, where transitions with indexes can consume
bound names of words.
%
Thus, we use canonical expressions in order to represent nominal
languages concretely.

%%% Local Variables:
%%% mode: latex
%%% TeX-master: "main"
%%% End:
