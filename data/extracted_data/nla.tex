In this section, we introduce our learning algorithm based on nominal
automata. Our teacher still answers two kinds of queries: membership
queries and equivalence queries regarding a target nominal regular
language.

\subsection{Preliminaries}
Before introducing our learning algorithm, some auxiliary notions are
necessary to give a concrete representation of nominal languages
and automata.
%
In fact, binders yield infinitely many equivalent representation of nominal words
due to alpha-conversion.
%
For instance, $\ll \aname. \aname \gg$ and
$\ll \aname[m].\aname[m] \gg$ are the same nominal word \emph{up-to}
renaming of their bound name.
%
We introduce \emph{canonical expressions} to give a finitary
representation of nominal regular languages.

\begin{definition}[Canonical expressions]\label{def:CE}
  Let $1 \leq n  \in \Nat$ a natural number and $ne$ a closed nominal
  regular expression.
  %
  The \emph{$n$-canonical representation} $\ce{n}{ne}$ of $ne$ is
  defined as follows
  \begin{itemize}
  \item $ne \in \{\emptyword,\emptyset\}\cup \Sigma \implies \ce{n}{ne} =ne$
  \item $\ce{n}{ne+ne'} =\ce{n}{ne}+\ce{n}{ne'}$
  \item $\ce{n}{ne\cdot ne'} =\ce{n}{ne}\cdot \ce{n}{ne'}$,
  \item $\ce{n}{ne^*} =(\ce{n}{ne})^*$
  \item $ne=\langle \aname. ne'\rangle \implies \ce{n}{ne} =\langle n. \ce{n+1}{ne'[n/\aname]}\rangle$
  \end{itemize}
  where $ne'[n/\aname]$ is the capture-avoiding substitution of
  $\aname$ for $n$ in $ne'$.
  %
  The \emph{canonical representation} of $ne$ is the term
  $\ce 1 {ne}$.
\end{definition}
%
Note that the map $\ce{\_}{\_}$ does not change the structure of the
nominal regular expression $ne$.  Basically, $\ce \_ \_$ maps nominal
regular expressions to terms where names are concretely represented as
positive numbers.
\begin{example}
  Given $\Sigma=\{a,b\}$, we give some examples of canonical
  representations of nominal expressions.
  \begin{small}
  \begin{itemize}
  \item $aba$ is the canonical representations of itself; indeed $\ce{1}{aba} = aba$
  \item $\ce{1}{\langle \aname . a\, \aname \rangle} = \ce{1}{\langle \aname[m]. a\, \aname[m] \rangle}{}=\langle 1.a\,1 \rangle$
    is the canonical representation of both $\langle \aname.a\,\aname \rangle$ and $\langle \aname[m].a\,\aname[m] \rangle$
  \item the canonical representation of
    $ne = \langle \aname.a\,\aname \langle \aname[m].\aname\, b\, \aname[m]
    \rangle\rangle\langle \aname[m].\aname[m] \rangle$ is
    $\ce{1} {ne} = \langle 1.a1\ce{2}{(\langle \aname[m].1\, b \, \aname[m] \rangle)}\langle
    1.1\rangle \\=\langle 1.a\,1\langle 2.1\,b\,2 \rangle \rangle\langle 1.1
    \rangle$.
  \end{itemize}  
  \end{small}
  Note that the map $\ce{\_}{\_}$ replaces names with numbers so that
  alpha-equivalent expressions are mapped to the same term (second example above).
\end{example}
%
Canonical expressions are the linguistic counter part of the mechanism
used in the definition of $(\Sigma, \Nom_{fin})$-automaton give
in~\cite{Kurz0T13}, where transitions with indexes can consume
bound names of words.
%
Thus, we use canonical expressions in order to represent nominal
languages concretely.

%%% Local Variables:
%%% mode: latex
%%% TeX-master: "main"
%%% End:

             
Fix a nominal regular language $L$, in our algorithm, the learner asks for membership queries about legal
words. If a word $w$ is not legal, the learner marks it as
$\bot$ in the observation table. The membership query consists of a
legal word $w$ and it has the following possible answers:
\begin{itemize}
\item if $w\in L$, the answer is \quo{1},
\item if $w$ is a prefix of a word in $L$, the answer is \quo{$P$},
\item otherwise, the answer is \quo{0}.
\end{itemize}

As in Angluin's \lstar algorithm, only the teacher knows $L$.
%
Unlike in \lstar, the learner in our algorithm does not know the whole
alphabet $A_n$.
%
The learner knows $\Sigma$ initially and learns names via
counterexamples.
%
We will see that the learner knows the whole alphabet
when the algorithm terminates.
             % research questions here?
             % give a brief explanation about why we let learner learn the alphabet not know it.
             % give a brief how we reach the final version among many possibilities.
  \paragraph{Remark.}The answer \quo{P} is used for efficiency. In
  fact, the teacher could answer \quo{0} instead of \quo{P}. However,
  this would require the learner to ask more membership or equivalence
  queries.
  %
  This is confirmed by some experimental results that are not in scope of this paper.
  % A comparison is provided in Table~\ref{tb:comparison}.
\subsection{Nominal observation tables}
Observation tables are pivotal data structure to ensure the
algorithm's functionalities. A closed and consistent observation table
allows us to construct a minimal automaton.
%
We extend Angluin's observation tables to \emph{nominal observation
  tables}, \nOtable[s] for short.
\begin{definition}[\nOtable]\label{def:nOT}
A \emph{legal word} is a prefix of a nominal word; the \emph{depth}
  $\|w\|$ of a legal word $w$ is the highest number of nested binders
  in $w$.
  % 
  Let
  \[
    \Alp_0 = \Sigma
    \qquad\text{and}\qquad
    \Alp_n= \{\ll,\gg\} \cup \Sigma\cup \underline{n}
    \quad \text{ for } 0 < n \in \Nat
  \]
  A tuple $\Otable$ is an
  \emph{\nOtable} if
  \begin{itemize}
  \item{$S\subseteq {\Alp_n}^*$ is a prefix-closed set of legal strings, for all $s\in S$,$\reg{s}\leq n$, }
  \item{$E\subseteq {\Alp_n}^*$ is suffix-closed,}
 % \item{$lv:\reg{s}\rightarrow \mathbb{N}$,}  
  \item{$T:(S\cup S\cdot {\Alp_n})\cdot E\rightarrow \labelsinT$.}
  \end{itemize}
\end{definition}
As in Angluin's definition, an \nOtable $\Otable$ consists of rows
labelled by legal words in $S\cup S\cdot \Alp_n$ and columns labelled
by words in $E$:
\begin{gather*}
  \row {}:{}   (S\cup S\cdot \Alp_n) \to (E \to \labelsinT)\\
  \row(s)(e)= T(s\cdot e)
\end{gather*}
In order to reflect the layers of nominal automata, we use $\reg{\_}$ to distinguish rows. Therefore, we need the following auxiliary notion of equivalence of rows:
in an \nOtable $\Otable$, for all $ s, s'\in S\cup S\cdot \Alp_n$, $$\row(s)\doteq\row(s')\ \Longleftrightarrow\ \row(s)=\row(s')\  and\ \reg{s}=\reg{s'}.$$ 
Accordingly, the definition of closed and consistent table changes as follows.
\begin{definition}[Closed and Consistent Tables]
  An \emph{\nOtable} $\Otable$ is \emph{closed} when
  $$\forall s'\in S\cdot {\Alp_n}.\exists s\in S.\ \ \row(s')\doteq\row(s).$$ 
  An \emph{\nOtable} $\Otable$ is \emph{consistent} when $$\forall \alpha\in {\Alp_n}.\forall s,s'\in S\ \ \row(s)\doteq\row(s')\implies \row(s\alpha)\doteq\row(s'\alpha). $$
\end{definition}
\subsection{From \nOtable[s] to nominal automata}
Analogously to Angluin's theory, closed and consistent \nOtable[s]
correspond to deterministic finite nominal automata.
\begin{definition}
The ($\Sigma,\mathcal{N}_{fin}$)-automaton $M=(Q,q_0,F,\delta)$ associated with a closed and consistent \emph{\nOtable} $\Otable$ is defined as 
\begin{itemize}
\item{$\Sigma={\Alp_n}\setminus\{\{\ll,\gg\}\cup\underline{n}\}$, $\mathcal{N}_{fin}=\underline{n}$,}
\item{a set of states $Q=\{(\row(s),\reg{s})\mid s\in S \}$ with a map $\Mreg{\_}$, and $\Mreg{q}=\reg{s}$ for each $q=(\row(s),\reg{s})\in Q$,}
\item{an initial state $q_0=(\row(\emptystring),\reg{\emptystring})$,}
\item{a set of final states $F=\{(\row(s),\reg{s})\mid \row(s)(\emptystring)=1,\reg{s}=0\ and\ s\in S \}$, }
%\item{for each $q\in Q$, we have a set $\delta(q,\alpha)\subseteq Q$ over ; for all $q'\in \delta(q,\alpha)$ must hold: $q=\state{s}, s\in S\implies s\cdot \alpha\in S\cup S\cdot A_n$. A transition is a triple $(q,\alpha,q')$ such that $q'\in \delta(q,\alpha)$.}
\item{A transition function is a partial function $\delta:Q\times {\Alp_n}\rightarrow Q$: for all $s\in S,\alpha\in {\Alp_n}$,
$\delta((\row(s),\reg{s}),\alpha)=
                   (\row(s\alpha),\reg{s\alpha})$ if $s\alpha\in S\cup S\cdot {\Alp_n}$.
}
\end{itemize}
\end{definition}


Accordingly, we define a partial function $\delta^\ast: Q\times {\Alp_n}^* \to Q$ inductively as follows
\begin{align*}
  \delta^\ast(q,\emptyword) &= q\\
  \delta^\ast(q,aw) &= \delta^\ast(\delta(q,a),w)
\end{align*}
for all $a\in \Alp_n,\ w\in {\Alp_n}^*$, $q\in
Q$.  Note that $\delta^*(q,a)=\delta^*(q,a\cdot
\emptyword)=\delta^*(\delta(q,a),
\emptyword)=\delta(q,a)$.
\begin{lemma}\label{lem:nTabledeltastar}
  Let $M=(Q,q_0,F,\delta)$ be the automaton associated with a closed and consistent \nOtable $\Otable$. Suppose $w,u\in{\Alp_n}^*$. We have  $\delta^\ast(q,w\cdot u)=\delta^\ast(\delta^\ast(q,w), u)$ for all $q\in Q$.
\end{lemma}

\begin{proof}
  By induction on length of $w$.
\end{proof}

\begin{theorem}\label{thm:nTabledeltastar}
  Assume that  $M=(Q,q_0,F,\delta)$ is the automaton associated with a closed and consistent \nOtable $\Otable$.
  \begin{itemize}
  \item
    For all $w$ in $S\cup S\cdot {\Alp_n}$, $\delta^\ast(q_0,w)=\state{w}$.
  \item
    For all $w$ in $S\cup S\cdot {\Alp_n}$ and $u$ in $E$, $\delta^\ast(q_0,w\cdot u)$ in $F$ if and only if $$\row(w)(u)=1.$$
  \end{itemize}
\end{theorem}
\begin{proof}
  Let $w=w'a$ in $S\cup S\cdot {\Alp_n}$ and $u=a\cdot u'$ in $E$.\\
  \\
  Since $S$ is prefix-closed, all prefixes of $w$ are in $S$, that is, $w'$ is in $S$. We know:
  \begin{align*}
    \delta^\ast(q_0,w)&=\delta^\ast(q_0,w'a)\\
                      &=\delta^\ast(\delta^\ast(q_0,w'),a)&& \textrm{by Lemma~\ref{lem:nTabledeltastar}}\\
                      &=\delta^\ast(\state{w'},a)&& \textrm{by induction hypothesis}\\
                      &=\delta(\state{w'},a) && \textrm{by the definition of $\delta^\ast$}\\
                      &=\state{w'a} && \textrm{by the definition of $\delta$}\\
                      &=\state{w}
  \end{align*}
  Since $E$ is suffix-closed, all suffixes of $u$ are in $E$. Depending on the length of $u$, we have two situations.
  \begin{itemize}
  \item When
    $u=\epsilon$,
    $\row(w)(u)=\row(w)(\emptyword)$ and
    $\delta^\ast(q_0,w\cdot
    u)=\delta^\ast(q_0,w)$. From preceding
    proof,
    $\delta^\ast(q_0,w)=\state{w}$. Because
    the table is closed, there is a $w'\in
    S$ such that $\row(w')\doteq
    \row(w)$. $\delta^\ast(q_0,w\cdot
    u)$ is in $F$ if and only if
    $\row(w')$ is in
    $F$ from the definition of
    $F$. Thus
    $\row(w')(\emptyword)=\row(w)(\emptyword)=1$,
    that is, $\row(w)(u)=1$.
  \item{Assume that when the length of $u'\in E$ is $n$, we have that for all $w$ in $S\cup S\cdot {\Alp_n}$ and $u$ in $E$, $\delta^\ast(q_0,w\cdot u)$ in $F$ if and only if $\row(w)(u)=1$. Let $u=au'$ and $u\in E$. Because the table is closed, there is a $w'\in S$ such that $\row(w')\doteq\row(w)$.
      \begin{align*}
        \delta^\ast(q_0,w\cdot u)
        &=\delta^\ast(\delta^\ast(q_0,w), u) && \textrm{by Lemma~\ref{lem:nTabledeltastar}}\\
        &=\delta^\ast(\state{w}, u) && \textrm{by preceding proof}\\
        &=\delta^\ast(\state{w'},u) && \textrm{since $\row(w')=\row(w)$}\\
        &=\delta^\ast(\state{w'},au') && \textrm{$u=au'$}\\
        &=\delta^\ast(\row(w'\cdot a),u') && \textrm{by closedness and definition of $\delta$}\\
        &=\delta^\ast(\delta^\ast(q_0,w'\cdot a),u')&& \textrm{by preceding proof}\\
        &=\delta^\ast(q_0,w'\cdot a\cdot u')
      \end{align*}
    }
  \end{itemize}
  By induction hypothesis on $u'$, $\delta^\ast(q_0,w'\cdot a\cdot
  u')$ is in $F$ if only if $\row(w'\cdot a)(
  u')=1$. Because $\row(w)\doteq \row(w')$ and $u=au'$, $\row(w'\cdot
  a)( u')=T(w'\cdot a\cdot u')=\row(w')( a\cdot
  u')=\row(w')(u)=\row(w)(u)$. Therefore $\delta^\ast(q_0,w\cdot
  u)$ in $F$ if and only if $\row(w)(u)=1$.
\end{proof}


We are now ready to introduce a learning algorithm for our nominal
automata.

\section{The \nlstar\ Algorithm}
%
We dubbed our algorithm \nlstar, after \emph{nominal} \lstar.
%
The algorithm is shown in Figure~\ref{LANA_a3}.
%
The learner in \nlstar is similar to the one in \lstar.
%
Basically, our learner modifies the initial \nOtable until it becomes
closed and consistent in the nominal sense (according to notions
introduced before).
%
When the current \nOtable $\Otable$ is closed and consistent, the
learner would ask the teacher if the automaton associated with $\Otable$
accepts the input language $L$.
%
If this is the case, the teacher will reply \quo{yes} and the learning process halts.
%
Otherwise, the learning process continues after the teacher has
produced a counterexample.

Because of the new definitions of closedness and consistency, we
refine some actions about checking closedness
(line~\ref{codeline:closedchanges}) and consistency
(line~\ref{codeline:findrowsconsistency}).
%
If the current \nOtable $\Otable$ is not closed, the learner finds a
row $s'$ such that for no $s\in S$ we have $\row(s') \doteq \row(s)$.
%
If $\Otable$ is not consistent, the learner finds a word $a\cdot e$,
with $a\in \Alp_n$ and $e\in E$, such that for some $s_1\in S$ and
$s_2\in S$ with $\row(s_1) \doteq \row(s_2)$, we have
$\row(s_1\cdot a)(e)\neq\row(s_2\cdot a)(e)$.

\begin{figure}
  \centering\small
  \begin{algorithmic}[1]
    \State{Initialisation: $S=\{\emptyword\}, E=\{\emptyword\}$, $n=0$}\label{codeline:init}
    \State{Asking for membership queries about $\emptyword$ and each $a \in \Alp_n$ build the initial observation table $\Otable$.}
    \Repeat
    \While{$\Otable$ is not closed or consistent}\label{codeline:while1}
    \If{$\Otable$ is not closed}\label{codeline:closedstatement}
    \State{\textbf{find} $s'\in S\cdot \Alp_n$ such that $\row(s)\doteq\row(s')$ is not satisfied for all $s\in S$  \;}\label{codeline:closedchanges}
    \State{\textbf{add} $s'$ into $S$\;}\label{codeline:addstate}
    % \State{\textbf{extend} $\Alp_n$ with $\reg{s}$ for all $s\in S$, $\reg{s}>0$.}\label{codeline:alphabetchanges}
    \State{\textbf{extend} $T$ to $(S\cup S\cdot \Alp_n)\cdot E$ using membership queries.}\label{codeline:closedextension}
    
    \EndIf
    \If{$\Otable$ is not consistent}\label{codeline:consiststatement}
    \State{\textbf{find} $s_1,s_2\in S$,$e\in E$ and $a\in \Alp_n$ such that \label{codeline:findrowsconsistency}
      $\row(s_1)\doteq\row(s_2)$ but $\row(s_1\cdot a)(e)\neq \row(s_2\cdot a)(e)$.}\label{codeline:consistentchanges
    }
    
    \State{\textbf{Add} $a\cdot e$ into $E$\;}\label{codeline:addE}
    %	\EndIf
    \State{\textbf{extend} $T$ to $(S\cup S\cdot \Alp_n)\cdot E$ using membership queries.}\label{codeline:consistentextension}
    \EndIf
    \EndWhile
    \State{Construct an automata $M$ associated to $\Otable$.}
    \State{Ask equivalence query about $M$.}
    \If{teacher replies a counterexample $c$}\label{codeline:replycounterexample}
    \State{add $c$ and all its prefixes into $S$.}\label{codeline:addcounterexample}
    %	\State{A set $temp$=GENERATENAMES(c)}
    %	\State{$S=S\cup temp$, also add all prefixes of $temp$'s elements into $S$ }
    %	\If{$c$ includes $\langle$}
    % \State{calculate $n$ the number of continuous occurrences of $\langle$ in $c$}
    % \State{Add $n$ into $A$}
    %	\EndIf
    \State{\textbf{extend} $\Alp_n$ with $\reg{s}$ for all $s\in S$, $\reg{s}>0$.}\label{codeline:alphabetchanges2}
    \State{\textbf{extend} $T$ to $(S\cup S\cdot \Alp_n)\cdot E$ using membership queries.}
    \EndIf
    \Until{teacher replies yes to $M$.}\label{codeline:replyyes}
    
    \State{Halt and output $M$.}
  \end{algorithmic}
  % \end{algorithm}
  \caption{The learner of Learning Algorithm for Nominal Automaton}\label{LANA_a3}	
\end{figure}
The main difference with respect to the algorithm of Angluin is that
the learner has partial knowledge of the alphabet.
%
The alphabet $\Alp_n$ is enlarged by adding names
(line~\ref{codeline:alphabetchanges2}) during the learning
process. More precisely, the learner expands the alphabet if the
counterexample requires to allocate fresh names.
%
When names are required, the operators of allocations and deallocations are added into $\Alp_n$. When the algorithm terminates, the learner's alphabet $\Alp_n$ is the
alphabet of the given language. Like in the original algorithm, our
algorithm terminates when the teacher replies \quo{yes} to an equivalence
query.

We now show that \nlstar\ is correct.
% 
That is, that eventually the teacher replies \quo{yes} to an equivalence query.
% 
In other word, \nlstar\ terminates with a \quo{yes} answer to an equivalence query.
% 
Hence, the automaton submitted in the query accepts the input language.

\begin{theorem}\label{thm:correct}
  The algorithm terminates, hence it is correct.
\end{theorem}
\begin{proof}
  We show that the if- and the while-statements terminate.
  %
  Let us consider the if-statements first.
  %
  It is easy to check that  closedness and
  consistency are decidable because these properties require just the inspection of the
  \nOtable (which is finite).
  %
  Hence, the if-statements starting at
  lines~\ref{codeline:closedstatement}
  and~\ref{codeline:consiststatement} never diverge because their
  guards do not diverge and their then-branch is a finite sequence of
  assignments:
  \begin{itemize}
  \item The if-statement for closedness
    (line~\ref{codeline:closedstatement}) terminates directly if the
    table is closed. Otherwise, in case of making a table closed, we
    find a row $s' \in S \cdot \Alp_n$ such that $\row(s')\doteq\row(s)$
    is not satisfied for all $s\in S$. The algorithm adds $s'$ into
    $S$. Since $\Alp_n$ is finite and bounded by $n$, the sets $S$ and
    $E$ are both finite. Thus, $S \cdot \Alp_n$ is a finite set and
    there are finitely many choices for $s'$. That is, 
    line~\ref{codeline:addstate} can only be executed finitely
    times. Besides, the content of rows is one of the permutations and
    combinations of $\labelsinT$ which also has finite
    possibilities. So we conclude that the branch terminates.
  \item Similarly, the if-statement for consistency
    (line~\ref{codeline:consiststatement}) terminates directly if the
    table is consistent. Otherwise, to make the table consistent the
    algorithm searches for two rows $s_1,s_2\in S$ satisfying the
    condition at line~\ref{codeline:findrowsconsistency}.
    %
    As in the previous case for closedness, to add elements into $E$
    there are only finitely many possibilities $s_1$, $s_2$, $a$, and
    $e$ (line~\ref{codeline:addE}). Thus, the branch of the
    if-statement terminates.
\end{itemize}
  Therefore, the while-statement (line~\ref{codeline:while1}) terminates in finite repetitions, since the algorithm makes a table closed and consistent in finite operations. Then, the algorithm will succeed in construct an automata $M$ associated to a closed and consistent table. Next, the learner asks for an equivalence query. The teacher replies a counterexample $c$ (line~\ref{codeline:replycounterexample}) or yes (line~\ref{codeline:replyyes}). It remains to prove that the learner only asks finitely many equivalence queries.

  Let $M$ be the nominal automaton associated to the current \nOtable
  $\Otable$. Assume that the equivalence query about $M$ fails. The
  teacher has to find a counterexample $c$; this is finitely
  computable since the nominal regular expressions are closed under
  the operations of Kleene algebra and under resource complementation.
  Hence, the if-statement on line~\ref{codeline:replycounterexample}
  goes the branch extending the table
  (line~\ref{codeline:addcounterexample}). Then, the algorithm will
  start a new loop (line~\ref{codeline:while1}) for the modified table
  by the counterexamples.
  %
  We can prove that a closed and consistent table builds a minimal
  finite automaton (omitted for space reasons).
  %
  A new automaton $M'$ will be constructed when the extended table
  $\Otable$ is closed and consistent. Since $M'$ handles the
  counterexample, $M'$ has more equivalent states to the minimal
  automaton accepted the given language, compared with $M$. Repeating
  this process, the automaton associated with a closed and consistent
  table has the same number of the minimal automaton which accepted
  the given language. Since the minimal automaton is unique, the two
  minimal automata are equal. The teacher relies yes to the automaton
  at such a point. Therefore, with respect to the number of the states
  of the minimal automaton accepted the given language, equivalence
  queries are finite and the algorithm terminates finally.
\end{proof}

In the following, we analyse the number of queries and the execution
time of \nlstar\ in the worst case. Let $M$ be the minimal automaton
accepting the given language and let $M$ have $\mathfrak{s}$ states.
% 
Let $\mathfrak{b}$ be a bound on the maximum length of the
counterexamples presented by the teacher.

From the Figure~\ref{LANA_a3} (line~\ref{codeline:init}), we know that $S$ and $E$ contain one element $\lambda$ initially. As the algorithm runs, it will add one element to $S$ when $\Otable$ is not closed (line~\ref{codeline:closedextension}). And it will add one element to $E$ when $\Otable$ is not consistent (line\ref{codeline:consistentextension}). For each counterexample of length at most $\mathfrak{b}$ presented by the teacher, the algorithm will add at most $\mathfrak{b}$ elements to $S$ (line~\ref{codeline:addcounterexample}).

Thus, the cardinality of $S$ is depends on $\mathfrak{s}$ and $\mathfrak{b}$. In detail, $S$ is at most $$1+(\mathfrak{s}-1)+\mathfrak{b}(\mathfrak{s}-1)=\mathfrak{s}+\mathfrak{b}(\mathfrak{s}-1)$$ because $\Otable$ can be not closed at most $\mathfrak{s}-1$ times. As the same as the teacher replies counterexamples at most $\mathfrak{s}-1$ times. And each time the teacher replies with a counterexample of length $\mathfrak{b}$, $S$ will be increased by at most $\mathfrak{b}$ elements. 

The cardinality of $E$ is at most $\mathfrak{s}$, because $\Otable$ can be not consistent at most $\mathfrak{s}-1$ times.   

The cardinality of $S\cdot \Alp_n$ could calculate from two parts: the cardinality of $S$ and the cardinality of $\Alp_n$. We already know the cardinality of $S$ is at most $\mathfrak{s}+\mathfrak{b}(\mathfrak{s}-1)$. As the definition of $\Alp_n$, let $\mathfrak{k}$ be the cardinality of $\Sigma$. Therefore the cardinality of $\Alp_n$ is $\mathfrak{k}+n+2$ and the cardinality of $S\cdot \Alp_n$ is $(\mathfrak{k}+n+2)(\mathfrak{s}+\mathfrak{b}(\mathfrak{s}-1))$ at most.

Therefore, the maximum cardinality of $(S\cup S\cdot \Alp_n)\cdot E$ is at most $$(\mathfrak{k}+n+2)(\mathfrak{s}+\mathfrak{b}(\mathfrak{s}-1))\mathfrak{s}=O((\mathfrak{k}+n)\mathfrak{b}\mathfrak{s}^2).$$ 
% The maximum length of any word in $(S\cup S\cdot \Alp_n)\cdot E$ is at most $$

The minimal automaton $M$ has $\mathfrak{s}$ states and the table $\Otable$ has one row initially. In the worst case, the table $\Otable$ adds only a distinguished row by every counterexample. The algorithm produces at most $\mathfrak{s}-1$  equivalence queries.

\paragraph{Running \nlstar: An Example}
Given a finite alphabet $\Sigma=\{a,b\}$, we have an example of learning a language $L$ representing as canonical nominal regular expression $cne=ab\langle 1^*\rangle$. 

In the first step, we initialize $S_1=\{\emptystring\}$, $E_1= \{\emptystring\}$, $n=0$ and ${\Alp_0}=\Sigma$, and construct $T_1$ as follows.\\
\textbf{Step 1}\\
\begin{minipage}{0.5\textwidth}
  \begin{center}
    $T_1$=
    \begin{tabular}{l|l|l}
      
      {$\reg{\_}$}  &{}  &      {$\emptystring$} \\
      \hline
      {0}  &  ${\emptystring}$ & {P} \\
      \hline
      {0}  & {$a$} & {P} \\
      {0}  &{$b$} & {0}\\
      
    \end{tabular}
  \end{center}
\end{minipage}
\begin{minipage}{0.5\textwidth}
  $(S_1,E_1,T_1,{\Alp_0})$ consistent?  There is only one row in $S$, thus the table is consistent.\\
  $(S_1,E_1,T_1,{\Alp_0})$ closed? No, $\row(b)\neq \row(\emptystring)$.\\
  So, $S_2\leftarrow S_1\cup \{b\}$ and we go to step 2.
\end{minipage}
\textbf{Step 2}\\
Let $S_2= S\cup \{b\}$ and $E_2=E$ and then construct a new observation table $(S_2,E_2,T_2,{\Alp_0})$ through membership queries.\\
\begin{minipage}{0.5\textwidth}
  \begin{center}
    $T_2$=
    \begin{tabular}{l|l|l}
      
      {$\reg{\_}$}  &{}  &      {$\emptystring$} \\
      \hline
      {0}  & ${\emptystring}$ & {P} \\
      {0}  &{$b$} & {0} \\
      \hline
      {0}  &{$a$} & {P}\\
      {0}  &{$aa$} & {0}\\
      {0}  &{$ab$} & {P}\\
    \end{tabular}
  \end{center}
\end{minipage}
\begin{minipage}{0.5\textwidth}
  $(S_2,E_2,T_2,{\Alp_0})$ closed? $\surd$ \\
  $(S_2,E_2,T_2,{\Alp_0})$ consistent? $\surd$\\
  Then, we compute the automaton $M$:
  
  \begin{tikzpicture}[>=stealth',shorten >=1pt,auto,node distance=15mm]
    \node[initial,state,inner sep=1pt,minimum size=0pt] (q0)      {$q_0$};
    \node[state,inner sep=1pt,minimum size=0pt]         (q1) [right of=q0]  {$q_1$};
    
    \path[->] (q0)  edge [bend left] node {$a$} (q1)
    edge [loop above] node {$b$} (q0)
    (q1) edge [bend left] node {$a$} (q0)
    edge [loop above] node {$b$} (q1);
    
  \end{tikzpicture}
  \\
  Teacher replies no and a counterexample, say, $ab\ll 1. \gg$. It is in $L$ not in $M$. And we go to step 3.
\end{minipage}
\textbf{Step 3}\\
Let $S_3\leftarrow S_2\cup \{a,ab, ab\ll 1.,ab\ll 1. \gg\}$, $E_3\leftarrow E_2$, and $n=1$, and then, the alphabet is extended to ${\Alp_1}=\Sigma\cup \underline{n}\cup\{\ll,\gg\}$. We should construct new observation tables sequentially through membership queries. Then we check the new table for closeness and consistency.\\
\begin{minipage}{0.5\textwidth}
  \begin{center}
    $T_3$=
    \begin{tabular}{l|l|l}
      
      {$\reg{\_}$}  &{}  &      {$\emptystring$}     \\
      \hline
      {0}  &${\emptystring}$ & {P} \\
      
      {0}  &{$b$} & {0} \\
      {0}  &{$a$} & {P}\\
      {0}  &{$ab$} & {P}\\
      {1}  & {$ab\ll 1.$} & {P} \\
      {0}  &{$ab\ll 1.\gg$} & {1} \\
      \hline
      {1}  &{$\ll 1.$} & {0} \\
      {0}  &{$a$} & {P} \\
      {0}  &{$ab\ll 1.\gg a$} & {0} \\
      {0}  &{$ab\ll 1.\gg b$} & {0} \\
      {1}  &{$ab\ll 1.1$} & {P} \\
      {1}  &{$ab\ll 1.a$} & {0} \\
      {$\cdots$}  &{$\cdots$} & {$\cdots$} \\
    \end{tabular}
  \end{center}
\end{minipage}
\begin{minipage}{0.5\textwidth}
  $(S_3,E_3,T_3,{\Alp}_1)$ consistent? \\
  No, $\row(ab)=\row(\emptystring)$ but $\row(ab\ll 1.)\neq \row(\ll 1.)$ . \\
  $(S_3,E_3,T_3,{\Alp}_1)$ closed? \\
   No,  $\row(\ll 1.)$ with $\reg{\ll 1.}=1$ has a fresh content. \\
\end{minipage}
\textbf{Step 4}\\
Let $S_4\leftarrow S_3\cup \{\ll 1. \}$,$E_4\leftarrow E_3\cup \{\ll 1.\}$, we should construct a new observation table $(S_4,E_4,T_4,{\Alp}_1)$ and check the new table for closeness and consistency.\\
\begin{minipage}{0.5\textwidth}
  \begin{center}
    $T_4$=
     \begin{tabular}{l|l|l|l}
    
    {$\reg{\_}$}  &{}  &      {$\emptystring$}   &      {$\ll 1.$} \\
    \hline
    {0}  &${\emptystring}$ & {P} & {0}\\
    
    {0}  &{$b$} & {0} & {0}\\
    {0}  &{$a$} & {P}& {0}\\
    {0}  &{$ab$} & {P}& {P}\\
    {1}  &{$ab\ll 1.$} & {P} & {$\bot$}\\
    {0}  & {$ab\ll 1.\gg$} & {1} & {0}\\	
    {1}  &{$\ll 1.$} & {0}& {$\bot$}\\
    \hline
    {0}  &	{$ba$} & {0}& {0}\\
    {0}  &	{$bb$} & {0}& {0}\\
    {1}  &	{$ab\ll 1.1$} & {P} & {$\bot$}\\
    {1}  &{$ab\ll 1.a$} & {0}  & {$\bot$}\\
    {$\cdots$}  &{$\cdots$} & {$\cdots$} & {$\cdots$}\\
  \end{tabular}
  \end{center}
\end{minipage}
\begin{minipage}{0.5\textwidth}
Once the table is closed and consistent, we ask an equivalence query.\\
Finally, the teacher replies \quo{yes} to an equivalence query about. The learning progress terminates. The learner automaton is as below.\\
\end{minipage}
\begin{tikzpicture}[>=stealth',shorten >=1pt,auto,node distance=15mm]
  \node[initial,state,inner sep=1pt,minimum size=0pt] (q0)      {$q_0$};
  \node[state,inner sep=1pt,minimum size=0pt]         (q1) [right of=q0]  {$q_1$};
  \node[state,inner sep=1pt,minimum size=0pt]         (q2) [right of=q1]  {$q_2$};
  \node[state,inner sep=1pt,minimum size=0pt]         (q3) [above of=q2]  {$q_3$};
  \node[state,accepting,inner sep=1pt,minimum size=0pt]         (q4) [right of=q2]  {$q_4$};
  \node[state,inner sep=1pt,minimum size=0pt]         (q5) [right of=q4]  {$q_5$};
  \node[state,inner sep=1pt,minimum size=0pt]         (q6) [right of=q3]  {$q_6$};
  \path[->] (q0)  edge node {$a$} (q1)
  edge[ bend right]  node[below] {$b$} (q5)
  
  (q1) edge[ bend right] node[below] {$a$} (q5)
  edge node {$b$} (q2)
  (q2) edge[ bend right]  node[below] {$a,b$} (q5)
  edge [ bend left] node {$\ll$} (q3)
  (q3) edge node {$a,b$} (q6)
  edge [loop above] node {$1$} (q3)
  edge [above] node {$\gg$} (q4)
  (q4) edge [bend right] node[below] {$a,b$} (q5)
  (q5) edge [loop right] node {$a,b$} (q5)
  (q6) edge [loop above] node {$a,b,1$} (q6)
  edge [bend left] node {$\gg$} (q5);
\end{tikzpicture}	


%%% Local Variables:
%%% mode: latex
%%% TeX-master: "main"
%%% End:
