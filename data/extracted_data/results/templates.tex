% \begin{table*}[tb]
% \centering
% \caption{Dataset Information (P $\rightarrow$ prescriptive; D $\rightarrow$ descriptive; Seq. Gen. $\rightarrow$ Sequence Generator; M-T$\rightarrow$ Multi-task ) \label{tbl:dataset}}
% \vspace{-4mm}
% %\resizebox{\textwidth}{!}{%
% \begin{tabular}{@{}L{0.4cm}ccccL{6.4cm}L{6cm}@{}}
% %@{}C{1.7cm}C{1.7cm}C{1.7cm}C{1.7cm}C{1.7cm}p{4cm}p{4cm}@{}}

% %\begin{tabular}{@{}C{1.7cm}C{1.7cm}C{1.7cm}C{1.7cm}C{1.7cm}p{4cm}p{4cm}@{}}
% \toprule
%   & P/D &\multicolumn{3}{c}{ Templates} & Example Template Task & Examples Template Result \\
% Task &&\rotatebox[origin=c]{90}{\# Trained} &\rotatebox[origin=c]{90}{\# New} &\rotatebox[origin=c]{90}{\# Samples} & & \\ \midrule
% \rotatebox[origin=c]{90}{Assign} & P. & 17 & 2 & 2000 
% & \noindent
% \textit{\small Given inputs 'a' and 'b', take the nor of these and return the result in 'c'.} 
% & \texttt{assign c = !(a | b);}
% \\ \midrule

% \rotatebox[origin=c]{90}{Assign} & D & 3 & 1 & 4000 
% & \textit{\small  A house has three active-low alarm detector triggered sensors 'a', 'b', 'c'. Write combinatorial logic for a active-high light 'l' which activates when any of the detectors are triggered.}
% & \texttt{assign l = !(a \& b \& c); }
% \\  \midrule

% \rotatebox[origin=c]{90}{Register} & P & 9 & 2 & 3000 
% & \textit{\small Define a 4-bit register 'q' with input 'a' nand 'b', enable 'e' defined as 'b' xnor 'r', an asynchronous reset 'r', and a clock 'c'.} 
% & \todo{from comments below}
% % assign e = b ^ r;
% % reg q;
% % always @(posedge c or posedge r) begin
% % if(r) begin
% %  q <= 0;
% % end else if(e) begin 
% %  q <=!(a & b);
% % end
% % end;}
% \\ \midrule

% \rotatebox[origin=c]{90}{Register} & D & 3 & 1 & 4000 
% & \textit{\small Design the code for an alarm system. When the panic mode 'm' is selected (= 0) the alarm system 'a' should activate (= 1) and should only deactivate (= 0) when the active-low synchronous cancel button 'c' is selected (= 1).
% } 
% & \todo{from comments below}
% % //assume clock clk
% % reg a;
% % always @(posedge clk) begin
% % if(c) begin
% %  a <= 0;
% % end else if(!m) begin 
% %  a <= 1;
% % end
% % end
% \\ \midrule

% \rotatebox[origin=c]{90}{Seq. Gen.}. & P & 4 & 2 & 4000 
% & \textit{\small Define sequential code which will produce the repeating sequence [0, 1, 0] on output 'u'. It should advance on clock 'c' whenever enable 'e' is present, and a synchronous reset 'r' should reset the sequence back to the first element.
% } 
% & \todo{code in comments below, good luck}
% % enum {s0, s1, s2} state; 
% % reg u;
% % always @(posedge c) begin
% % if(s) begin
% %  state <= s0;
% %  u <= b0;
% % end else begin 
% % unique case (state) 
% % s0: if(e) begin
% %  state <= s1;
% %  u <= b0;
% % end 
% % s1: if(e) begin
% %  state <= s2;
% %  u <= b1;
% % end 
% % s2: if(e) begin
% %  state <= s0;
% %  u <= b0;
% % end 
% % endcase 
% % end
% \\ \midrule

% \rotatebox[origin=c]{90}{M-T} & P/D & N/A & N/A & 5250 
% & \todo{stretch over both columns, no example provided}\textit{\small(Select and concatenate 2-4 templates from the  Register / Assignment categories).} 
% & \\ \bottomrule
% \end{tabular}%
% %}
% \vspace{-4mm}
% \end{table*} 

\begin{table*}[tb]
\caption{Template-based Dataset Information. (pX $\rightarrow$ prescriptive; dX $\rightarrow$ descriptive; X is the task type) \label{tbl:dataset}}
\vspace{-4mm}
\small
\renewcommand{\arraystretch}{0.2}
% \resizebox{\textwidth}{!}{%
\begin{tabular}{@{}C{0.2cm}C{0.5cm}C{1cm}C{1cm}C{1.5cm}L{6cm}p{5.5cm}@{}}
\toprule
\multicolumn{2}{c}{Task}  & \# for Training & \# Non-Training & Samples / Template & Example of Task in English & Model Verilog \vspace{-1mm}\\ \midrule
\multirow{2}{*}{\rotatebox[origin=c]{90}{Assignment (a)\hspace{-2mm}}} & pa & 17 & 2 & 2000 & {\it Given inputs `a' and `b', take the nor of these and return the result in `c'.} & {\begin{lstlisting}[aboveskip=0pt,belowskip=-12pt,frame=none]
assign c = !(a | b);
\end{lstlisting}}\\ \cmidrule(r){2-7}
& da & 3 & 1 & 4000 & {\it A house has three active-low alarm detector triggered sensors `a', `b', `c'. Write combinatorial logic for a active-high light `l' which activates when any of the detectors are triggered.}  & {\begin{lstlisting}[aboveskip=-5pt,belowskip=-12pt,frame=none]
assign l = !(a & b & c);
\end{lstlisting}} \\ \midrule
\multirow{2}{*}{\rotatebox[origin=c]{90}{Register (r)}} & pr & 9 & 2 & 3000 & {\it Define a 4-bit register `q' with input `a' nand `b', enable `e' defined as `b' xnor `r', an asynchronous reset `r', and a clock `c'.} &  
{\begin{lstlisting}[aboveskip=-15pt,belowskip=-25pt,frame=none]
assign e = b ^ r; reg q;
always @(posedge c or posedge r) begin
if(r) begin  q <= 0; end 
else if(e) begin  q <=!(a & b); end
end;
\end{lstlisting}}
\\ \cmidrule(r){2-7}
& dr & 3 & 1 & 4000 & {\it Design the code for an alarm system. When the panic mode `m' is selected (= 0) the alarm system `a' should activate (= 1) and should only deactivate (= 0) when the active-low synchronous cancel button `c' is selected (= 1).} &  
{\begin{lstlisting}[aboveskip=-20pt,belowskip=-25pt,frame=none]
//assume clock clk
reg a;
always @(posedge clk) begin
if(c) begin  a <= 0; end 
else if(!m) begin  a <= 1; end
end
\end{lstlisting}}
\\ \midrule
\rotatebox[origin=c]{90}{Sequence Generator (g)\hspace{-1mm}} & pg & 4 & 2 & 4000 & {\it Define sequential code which will produce the repeating sequence [0, 1, 0] on output `u'. It should advance on clock `c' whenever enable `e' is present, and a synchronous reset `r' should reset the sequence back to the first element.} &  
{\begin{lstlisting}[aboveskip=-25pt,belowskip=-20pt,frame=none]
enum {s0, s1, s2} state; reg u;
always @(posedge c) begin
if(s) begin state <= s0; u <= b0; end 
else begin 
unique case (state) 
s0: if(e) begin state <= s1; u <= b0; end 
s1: if(e) begin state <= s2; u <= b1; end 
s2: if(e) begin state <= s0; u <= b0; end 
endcase 
end
\end{lstlisting}}
\\ \midrule
% & mt & N/A & N/A  & 5250 & \multicolumn{2}{c}{\it Select and concatenate 2-4 templates from the Register / Assignment categories).} \\ \bottomrule
\rotatebox[origin=c]{90}{Multi-task (M-T)\hspace{-6mm}}& -- & N/A & N/A  & 5250 & {\it Write a 6-bit register `ar' with input defined as `gv' modulo `lj', enable `q', synchronous reset `r' defined as `yxo' greater than or equal to `m', and clock `p'. A vault door has three active-low secret switch pressed sensors `et', `lz', `l'. Write combinatorial logic for a active-high lock `s' which opens when all of the switches are pressed. Write a 6-bit register `w' with input  `se' and `md', enable `mmx', synchronous reset `nc' defined as `tfs' greater than `w', and clock `xx'.} & 
{\begin{lstlisting}[aboveskip=-45pt,belowskip=-20pt,frame=none]
assign r = yxo >= m; reg [5:0] ar;
always @(posedge p) begin
 if(r) begin ar <= 0; end 
 else if(q) begin ar <= gv % lj; end
end
assign s = !(et | lz | l); 
assign nc = tfs > w; reg [5:0] w;
always @(posedge xx) begin
 if(nc) begin w <= 0; end 
 else if(mmx) begin w <= se & md; end
end
\end{lstlisting}}
\\ \bottomrule
\end{tabular}%
% }
\vspace{-3mm}
\end{table*}
